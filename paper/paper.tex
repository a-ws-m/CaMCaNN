%%%%%%%%%%%%%%%%%%%%%%%%%%%%%%%%%%%%%%%%%%%%%%%%%%%%%%%%%%%%%%%%%%%%%
%% This is a (brief) model paper using the achemso class
%% The document class accepts keyval options, which should include
%% the target journal and optionally the manuscript type. 
%%%%%%%%%%%%%%%%%%%%%%%%%%%%%%%%%%%%%%%%%%%%%%%%%%%%%%%%%%%%%%%%%%%%%
\documentclass[journal=jcisd8,manuscript=article]{achemso}

%%%%%%%%%%%%%%%%%%%%%%%%%%%%%%%%%%%%%%%%%%%%%%%%%%%%%%%%%%%%%%%%%%%%%
%% Place any additional packages needed here.  Only include packages
%% which are essential, to avoid problems later. Do NOT use any
%% packages which require e-TeX (for example etoolbox): the e-TeX
%% extensions are not currently available on the ACS conversion
%% servers.
%%%%%%%%%%%%%%%%%%%%%%%%%%%%%%%%%%%%%%%%%%%%%%%%%%%%%%%%%%%%%%%%%%%%%
\usepackage{amsmath}
\usepackage{graphicx}
\usepackage{siunitx}
\usepackage{chemfig}
\usepackage[version=4]{mhchem}
\usepackage{booktabs}
\usepackage{subcaption}

%%%%%%%%%%%%%%%%%%%%%%%%%%%%%%%%%%%%%%%%%%%%%%%%%%%%%%%%%%%%%%%%%%%%%
%% If issues arise when submitting your manuscript, you may want to
%% un-comment the next line.  This provides information on the
%% version of every file you have used.
%%%%%%%%%%%%%%%%%%%%%%%%%%%%%%%%%%%%%%%%%%%%%%%%%%%%%%%%%%%%%%%%%%%%%
%%\listfiles

%%%%%%%%%%%%%%%%%%%%%%%%%%%%%%%%%%%%%%%%%%%%%%%%%%%%%%%%%%%%%%%%%%%%%
%% Place any additional macros here.  Please use \newcommand* where
%% possible, and avoid layout-changing macros (which are not used
%% when typesetting).
%%%%%%%%%%%%%%%%%%%%%%%%%%%%%%%%%%%%%%%%%%%%%%%%%%%%%%%%%%%%%%%%%%%%%
\setchemfig{angle increment=30,atom sep=2em}

%%%%%%%%%%%%%%%%%%%%%%%%%%%%%%%%%%%%%%%%%%%%%%%%%%%%%%%%%%%%%%%%%%%%%
%% Meta-data block
%% ---------------
%% Each author should be given as a separate \author command.
%%
%% Corresponding authors should have an e-mail given after the author
%% name as an \email command. Phone and fax numbers can be given
%% using \phone and \fax, respectively; this information is optional.
%%
%% The affiliation of authors is given after the authors; each
%% \affiliation command applies to all preceding authors not already
%% assigned an affiliation.
%%
%% The affiliation takes an option argument for the short name.  This
%% will typically be something like "University of Somewhere".
%%
%% The \altaffiliation macro should be used for new address, etc.
%% On the other hand, \alsoaffiliation is used on a per author basis
%% when authors are associated with multiple institutions.
%%%%%%%%%%%%%%%%%%%%%%%%%%%%%%%%%%%%%%%%%%%%%%%%%%%%%%%%%%%%%%%%%%%%%
\author{Alexander Moriarty}
\email{alexander.moriarty.21@ucl.ac.uk}
\author{Takeshi Kobayashi}
\author{Matteo Salvalaglio}
\author{Panagiota Angeli}
\author{Alberto Striolo}
\affiliation[UCL]{Department of Chemical Engineering, University College London, UK}
\alsoaffiliation[UO]{Gallolgy College of Engineering, University of Oklahoma, USA}
\author{Ian McRobbie}
\affiliation[Innospec]{Senior Vice President, Research and Technology, Innospec Ltd., Ellesmere Port, UK}

%%%%%%%%%%%%%%%%%%%%%%%%%%%%%%%%%%%%%%%%%%%%%%%%%%%%%%%%%%%%%%%%%%%%%
%% The document title should be given as usual. Some journals require
%% a running title from the author: this should be supplied as an
%% optional argument to \title.
%%%%%%%%%%%%%%%%%%%%%%%%%%%%%%%%%%%%%%%%%%%%%%%%%%%%%%%%%%%%%%%%%%%%%
\title{Exploring the Applicability Domain of Predictive Models for Critical Micelle Concentrations using Graph Neural Networks and Gaussian Processes}

%%%%%%%%%%%%%%%%%%%%%%%%%%%%%%%%%%%%%%%%%%%%%%%%%%%%%%%%%%%%%%%%%%%%%
%% Some journals require a list of abbreviations or keywords to be
%% supplied. These should be set up here, and will be printed after
%% the title and author information, if needed.
%%%%%%%%%%%%%%%%%%%%%%%%%%%%%%%%%%%%%%%%%%%%%%%%%%%%%%%%%%%%%%%%%%%%%
\abbreviations{CMC,ML,GNN,GP}
\keywords{Critical micelle concentration, machine learning, molecular modeling, molecules, neural networks, uncertainty quantification}

%%%%%%%%%%%%%%%%%%%%%%%%%%%%%%%%%%%%%%%%%%%%%%%%%%%%%%%%%%%%%%%%%%%%%
%% The manuscript does not need to include \maketitle, which is
%% executed automatically.
%%%%%%%%%%%%%%%%%%%%%%%%%%%%%%%%%%%%%%%%%%%%%%%%%%%%%%%%%%%%%%%%%%%%%
\begin{document}

%%%%%%%%%%%%%%%%%%%%%%%%%%%%%%%%%%%%%%%%%%%%%%%%%%%%%%%%%%%%%%%%%%%%%
%% The "tocentry" environment can be used to create an entry for the
%% graphical table of contents. It is given here as some journals
%% require that it is printed as part of the abstract page. It will
%% be automatically moved as appropriate.
%%%%%%%%%%%%%%%%%%%%%%%%%%%%%%%%%%%%%%%%%%%%%%%%%%%%%%%%%%%%%%%%%%%%%
\begin{tocentry}

    I'm looking for ideas for the TOC graphical entry -- common practice seems
    to be to draw a really abstract picture of a molecule going through a
    network, but that doesn't seem like the best thing to do here because I'm
    specifically comparing against other models.

\end{tocentry}

%%%%%%%%%%%%%%%%%%%%%%%%%%%%%%%%%%%%%%%%%%%%%%%%%%%%%%%%%%%%%%%%%%%%%
%% The abstract environment will automatically gobble the contents
%% if an abstract is not used by the target journal.
%%%%%%%%%%%%%%%%%%%%%%%%%%%%%%%%%%%%%%%%%%%%%%%%%%%%%%%%%%%%%%%%%%%%%
\begin{abstract}
    This paper presents a novel approach to predicting critical micelle
    concentrations (CMCs) using graph neural networks (GNNs) augmented with
    Gaussian Processes (GPs). The proposed model uses learned latent space
    representations of molecules to predict CMCs and estimate uncertainties. The
    performance of the model on a dataset containing nonionic, cationic, anionic
    and zwitterionic molecules, is compared against a linear model that works
    with extended-connectivity fingerprints (ECFPs). Our results demonstrate
    that the GNN-based model performs slightly better than the linear ECFP
    model, when there is enough well-balanced training data, and achieves
    predictive accuracy that is comparable to published models that were
    evaluated on a smaller range of surfactant chemistries. But the main benefit
    of the GNN model is the ability to provide uncertainty estimates. We
    illustrate the applicability domain of our model using a visualisation of
    the latent space, which helps identify molecules likely to have erroneous
    predictions. The proposed approach can help in the design of new surfactants
    and provide valuable insights into the molecular properties that influence
    CMCs.
\end{abstract}

%%%%%%%%%%%%%%%%%%%%%%%%%%%%%%%%%%%%%%%%%%%%%%%%%%%%%%%%%%%%%%%%%%%%%
%% Start the main part of the manuscript here.
%%%%%%%%%%%%%%%%%%%%%%%%%%%%%%%%%%%%%%%%%%%%%%%%%%%%%%%%%%%%%%%%%%%%%
\section{Introduction}

The critical micelle concentration (CMC) of a surfactant defines the concentration above which the surfactant monomers self-assemble in solution to form micelles. It is an important property because the formation of micelles affects many interfacial phenomena \cite{rosenSurfactantsInterfacialPhenomena2012} micelles can encapsulate hydrophobic molecules to form useful complexes, but the formation of micelles also inhibits certain processes, like the aggregation of surfactants at interfaces in order to reduce surface tension.

Perhaps the most well-established predictor for CMC, $X_{cmc}$, is the Stauff-Klevens relationship, first published in \citeyear{klevensStructureAggregationDilate1953} \cite{klevensStructureAggregationDilate1953}. It formalised the observation that CMC decreases exponentially with an increase in the number of carbons in the hydrocarbon tail, $n_c$:

\begin{equation}
    \label{eq:klevens}
    \log X_{cmc} = A - Bn_c, \quad B > 0
\end{equation}

where $A$ and $B$ are empirical constants that depend on the temperature and the homologous series, i.e. the headgroup. The model is simple yet accurate, and it is easily interpretable: to reduce CMC, it is sufficient to extend the surfactant's hydrocarbon tail thus defining an easy-to-apply qualitative heuristic. Its drawback as a predictive model is its very limited applicability domain; each set of parameters is only applicable to surfactants with a specific headgroup and a linear carbon tail.
One of the goals of the quantitative structure-property relationship (QSPR) development is to produce models that are general so that we can apply them to a diverse range of compounds, design novel molecules with target properties, and
interpret the models' results to glean chemical insights.

There have been a wealth of investigations into making more general models for
CMC prediction; here, a brief review of some diverse and promising approaches
for predicting CMCs of aqueous, single-surfactant systems will be given.

\citet{puvvadaMolecularThermodynamicApproach1990} derived a phenomenological
model for studying aqueous nonionic surfactant systems that enabled prediction
of CMC and other properties across a range of temperatures. They developed a
model for the free energy of micellization, from which the CMC can be
calculated. Their model was parameterized by three properties of the surfactant
molecule:
\begin{itemize}
    \item The tail length, defined as the number of carbon atoms.
    \item The average cross-sectional area of the headgroup, which controls the
          steric contribution to the free energy. This must be estimated.
    \item The Tolman length of the tail, which approximates the thickness of an
          `interaction region' around the tail
          \cite{demiguelGibbsThermodynamicsSurface2021}. This must also be
          estimated.
\end{itemize}

The authors developed a method to estimate parameters for linear, nonionic,
polyoxyethylene alkyl ether surfactants. The model attained a
root-mean-squared error (RMSE) of approximately \SI{0.14}{\log \micro M} for the
group \ce{C10E_i}, where $i \in (3, 6)$, and \SI{0.21}{\log \micro M} for the
group \ce{C12E_j}, where $j \in (3, 8)$. However, the error is much larger for
other systems, like \ce{C8E6}. The authors ascribed this inaccuracy to their
prediction that the micelles in these systems would not exhibit much anisotropic
growth; their model is better suited for cylindrical or disk-like (bilayer)
micelles.

The connection this model established between a small set of physically
meaningful properties is extremely useful, especially because it does not
explicitly require fitting to any experimental data. However, the procedures
described for estimating the Tolman length are only applicable to linear
hydrocarbon chains, not branched nor heterogeneous tail groups. Estimating the
average cross-sectional area of the head group may also not be trivial.

Greater generalizability and better accuracy can be realized by considering the
process of aggregation by identifying the behavior of individual atoms, or small
groups thereof. Simulation approaches can model the interaction of these units
with each other and derive the potential energy of a configuration
\cite{frenkelUnderstandingMolecularSimulation2001,
    joshiReviewAdvancementsCoarsegrained2021,
    filipeMolecularDynamicsSimulations2022}. For example, molecular dynamics (MD)
simulations treat individual atoms, in the all-atomistic (AA) approach, or
groups of atoms, in the coarse-grained (CG) approach, as particles in a box that
interact with each other. This allows the particles' movement to be simulated by
integrating the equations of motion.

For example, \citet{jorgeMolecularDynamicsSimulation2008} used an AA approach to
simulate the self-assembly of \textit{n}-decyltrimethylammonium bromide. They
then estimated the CMC by considering the concentration of `free' surfactants,
i.e. surfactants that were not in micelles, which they defined as an aggregate
containing five or more surfactants. However,
\citet{jusufiExplicitImplicitSolventSimulations2015} criticized the free
surfactant concentration approach for modelling CMC of ionic surfactants in
general. They note that free surfactant concentration above the CMC is highly
dependent on the choice of overall surfactant concentration, especially for
ionic surfactants, which necessitates careful extrapolation to accurately
determine CMCs.

Coarse-graining, which groups atoms into beads, makes simulating longer time
scales accessible \cite{fitzgeraldMultiscaleModelingNanomaterials2015,
khedrQuantificationOstwaldRipening2019,
bochicchioDynamicsSupramolecularPolymer2017, andersonMicelleFormationAlkyl2018}.
For example, \citet{khedrDPDParametersEstimation2018} used dissipative particle
dynamics (DPD) to model the CMCs of two polyoxyethylene octyl ether surfactants
(\ce{C8E_n}). Rather than using the free surfactant concentration by itself, the
authors calculated the volume fraction of free surfactant in the accessible
component of the aqueous phase. This approach accounts for the effective
reduction in accessible volume due to the occupancy of micellar aggregates
\cite{santosDeterminationCriticalMicelle2016}. With this approach, they
predicted CMCs for \ce{C8E1} and \ce{C8E9}, both with errors of approximately
\SI{0.03}{\micro M}. Despite the high accuracy of the approach, there is a
relative lack of validated DPD interaction parameters for different beads across
a range of temperatures \cite{nivon-ramirezCriticalMicelleConcentration2022}.

% Coarse-graining, which groups atoms into beads, makes simulating longer time
% scales accessible \cite{fitzgeraldMultiscaleModelingNanomaterials2015}. For
% example, \citet{vishnyakovPredictionCriticalMicelle2013} used dissipative
% particle dynamics (DPD) to model the CMCs of \ce{C8E8},
% dodecyldimethylamineoxide (DDAO) and
% \textit{N}-decanoyl-\textit{N}-methyl-\textsc{D}-glucamide (MEGA-10). They
% proposed a methodology for obtaining parameters to describe the bead
% interactions based on known infinite dilution activity coefficients,
% $\gamma_\infty$. Their calculated CMC for \ce{C8E8} had an error relative to the
% average experimental value equal to \SI{0.07}{\log \micro M}, for DDAO it was
% \SI{0.06}{\log \micro M}, and for MEGA-10 it was \SI{0.06}{\log \micro M}.

% Although this approach can produce extremely accurate results, it is limited to
% compounds that can be parameterized by a known $\gamma_\infty$. Furthermore, the
% authors do not discuss parameterizing ionic surfactants. Despite their high
% computational cost, both AA and CG simulation approaches can offer very deep
% insight into the processes occurring at a molecular level; their results yield a
% precise description of the arrangement of molecules in a system, their shape and
% aggregation number.

Another approach is the conductor-like screening model (COSMO), which decomposes
the problem by treating a molecule as a cavity with a charged surface in a
solvent that acts as a dielectric continuum \cite{klamtCOSMONewApproach1993}.
The cavity's surface is described by the solvent-accessible surface of the
molecule. The geometry of this surface combined with a segment-wise description
of its polarizing charges can be mapped using density functional theory (DFT).

COSMO for realistic solvation (COSMO-RS) adapts the model for more complex types
of solvent \cite{klamtCOSMORSAlternativeSimulation2010}. Solvents only act like
a dielectric continuum when they are capable of perfectly screening the
COSMO-surface of a solute. COSMO-RS uses statistical mechanics to determine the
probability distribution describing how surface charge densities align between
two molecules. This allows chemical potentials to be determined.

\citet{turchiFirstprinciplesPredictionCritical2022} used COSMO-RS to predict
CMCs by treating a micelle as a separate phase and then considering the
two-phase equilibrium between the micelle and an aqueous phase containing free
surfactants. They modelled the micellar `phase' using two strategies. Their
first strategy was to treat the micellar phase as being equivalent to a bulk,
homogeneous phase of surfactant. The CMC could then be determined by the
equilibrium surfactant concentration in the aqueous phase. The authors argued
that this approximation is more valid as the difference in polarity between the
head and tail of a surfactant is reduced, which was the case for the majority of
nonionic surfactants they considered.

Their second strategy was to consider the micellar phase as a bulk, homogeneous
phase of an oil, whose chemistry was analogous to that of the surfactant's tail
group. They then implemented an iterative procedure to calculate the interfacial
tension (IFT) between the oil and aqueous surfactant phases at different
concentrations of surfactant. The concentration at which the IFT is zero yields
the CMC prediction. The premise of this approach is that the oil phase is
representative of a micelle's interior, which is particularly true as the
interactions between head and tail groups become more unfavorable. This is the
case for highly polar head groups: primarily for ionic surfactants.

The authors recommended applying both strategies and using the lower result as
the CMC prediction. They attained an RMSE of \SI{0.81}{\log \micro M} on a
dataset of 24 surfactants, containing a mix of ionic, nonionic and zwitterionic
surfactants. It is notable that the technique can be applied across all classes
of surfactants.

Other approaches have extended COSMO-RS to explicitly account for the internal
structure of micelles, such as COSMOmic, which treats a micelle as being made of
concentric layers that each have their own surface charge profiles
\cite{klamtCOSMOmicMechanisticApproach2008}. For example,
\citet{jakobtorweihenPredictingCriticalMicelle2017} calculated CMCs using
COSMOmic by first performing MD simulations to attain the layer-wise atomic
distributions. The authors then predicted the CMCs of several polyoxyethylene
alkyl ethers by determining the partition coefficients of inserting the
respective surfactant monomer into a micelle.

COSMOplex is a recent extension of COSMOmic that removes the need to perform and
initial MD simulation to determine the micellar structure
\cite{klamtCOSMOplexSelfconsistentSimulation2019}. Instead, it optimizes the
micellar structure using a self-consistent approach, which iteratively yields
new estimates for the layer-wise charge distributions. The authors predicted the
CMCs of 10 nonionic surfactants with varied head and tail group chemistries,
achieving an RMSE of \SI{0.86}{\log \micro M}.

Although COSMO techniques are promising,
\citet{herbertDielectricContinuumMethods2021} note that many modern extensions
are only available in the proprietary software package \textsc{COSMOTHERM}
\cite{eckertFastSolventScreening2002}.

Another approach to predict CMCs is to use an equation of state. For example,
\citet{liStudiesUNIQUACSAFT1998} applied a segment-based UNIQUAC model
(s-UNIQUAC) and a SAFT equation of state to predict CMCs of linear
polyoxyethylene alkyl ethers by first deriving expressions for the activity
coefficient of a surfactant in water.

\citet{chengCorrelationCriticalMicelle2005} compared the performance of several
models on a large dataset: the polymer-NRTL model
\cite{liStudiesUNIQUACSAFT1998}, a UNIFAC model
\cite{voutsasPredictionCriticalMicelle2001} and a modified Aranovich and Donohue
(m-AD) model \cite{chengCorrelationCriticalMicelle2005}. The predictive
performance of the models on the molecular series \ce{C_nE6}, \ce{C_nE8},
\ce{C_nE9}, \ce{C10E_n} and \ce{C12E_n} were compared, and the resulting RMSEs
are summarized in Table \ref{tab:segment-methods}. The models all have a
reasonably good accuracy, but the SAFT model is particularly good.

\begin{table}
    \caption{Comparison of the RMSEs of selected models on polyoxyethylene alkyl
        ethers' CMCs. The best RMSE is underlined. Data from
        \citet{chengCorrelationCriticalMicelle2005}.}
    \label{tab:segment-methods}
    \begin{tabular}{lr}
        \toprule
        \multicolumn{1}{c}{Model} & \multicolumn{1}{c}{RMSE (\si{\log \micro M})} \\\midrule
        p-NRTL                    & 0.18                                          \\
        s-UNIQUAC                 & 0.14                                          \\
        SAFT                      & \underline{0.06}                              \\
        UNIFAC                    & 0.14                                          \\
        m-AD                      & 0.11                                          \\\bottomrule
    \end{tabular}
\end{table}

Segment-based semi-empirical methods are very promising for predicting CMCs
within a class of surfactants. Their major drawback is that they are only
applicable to molecules that can be decomposed into segments that have trained
parameters. In addition, they must respect the limitations of the theories they
are based upon.

Finally, purely empirical methods have a very heavy reliance on data abundance.
Empirical methods offer a way of making predictions even when a unified theory
is lacking or computationally too demanding. However, without an underlying
theory, their limitations are not well defined and it is possible for the model
to `learn' trends that contradict scientific intuition.

Empirical QSPR methods require validation to determine their reliability and
applicability domain
\cite{veerasamyValidationQSARModelsstrategies2011,tropshaBestPracticesQSAR2010,leonardSelectionTrainingTest2006};
the performance metrics during optimization are not a reliable indicator of
generality, or the performance on new molecules. This is often achieved by
partitioning the available data into \emph{training} and \emph{test} subsets;
the former is used for optimizing the model's parameters, the latter is `hidden'
from the model until training is complete, and the prediction metrics on the
test data indicate how the model can be expected to perform in general. The test
set should span the chemical space in which the model is intended to be applied
\cite{leonardSelectionTrainingTest2006}.

Empirical QSPR models can be used to design novel molecules with target
properties
\cite{gantzerInverseQSPRNovoDesign2020,bolboacaMolecularDesignQSARs2013} and are
interpretable \cite{zefirovFragmentalApproachQSPR2002}, meaning that they can be
analyzed to obtain chemical insights.

\citet{matteiModelingCriticalMicelle2013} extended the Marrero and Gani
group-contribution method \cite{ganiAutomaticCreationMissing2005} to predict the
CMCs of 150 nonionic surfactants. The descriptors are the number of each group
present in a molecule. In the original method
\cite{ganiAutomaticCreationMissing2005}, different `orders' of groups were
identified; the first-order groups are forbidden from overlapping with one
another and they are formulated so that any molecule of interest can be
described using these groups exclusively. Higher order groups distinguish
polyfunctional molecules and isomers \cite{ganiAutomaticCreationMissing2005}.

\citet{matteiModelingCriticalMicelle2013} introduced third-order groups to
improve their model's accuracy by analyzing the molecules with the highest
prediction errors after training an initial model with the first- and
second-order groups from a prior work \cite{ganiAutomaticCreationMissing2005}.
This is an example of \emph{feature selection}, whereby the set of descriptors
is expanded or contracted to adapt to the problem
\cite{liFeatureSelectionData2017,guyonIntroductionVariableFeature2003}.

The authors randomly selected 30 compounds as a test dataset, achieving a RMSE
of \SI{0.13}{\log \micro M}. The model is remarkably accurate and boasts high
interpretability: the fitted contributions of each group describe their effect
on the CMC quantitatively, and the existence of higher order polyfunctional
groups with large contributions implies that their constituent functional groups
have a significant interaction with each other that affects the CMC. However, it
may be difficult to determine whether a new molecule is within the applicability
domain of the model, particularly because positional isomers are not necessarily
distinguished from each other using the group representation.

Recently, an approach based on graph neural networks (GNNs) has produced highly
accurate predictions whilst being applicable to nonionic, cationic, anionic and
zwitterionic surfactants \cite{qinPredictingCriticalMicelle2021}. Neural
networks have many trainable parameters and a complex functional form. This
ensures their versatility as universal approximators but makes them highly
susceptible to overfitting \cite{bejaniSystematicReviewOverfitting2021}. Neural
networks potentially boast the largest applicability domain (for a single set of
trained parameters) of any model discussed previously.

GNN approaches operate on molecular graphs, which are characterized by atomic
nodes whose edges represent bonds. Each operation on this graph considers just
the local environment of an atom, i.e. the atoms that can be reached by
traversing a single bond, but by stacking these operations in sequence, the size
of the environment that is considered increases. In this sense the model is
similar to a group contribution approach, although the groups are determined by
walking $r$ steps along bonds from every atom in the molecule, where $r$ is
equal to the number of subsequent graph operations, so that every group
overlaps. Furthermore, each `contribution' is non-linear, and the number of
contributions is always equal to the number of atoms in a molecule. Because of
their versatility, GNNs have been applied across a plethora of computational
chemistry tasks, from molecular property prediction \cite{
leeTransferLearningMaterials2020, louisGraphConvolutionalNeural2020,
rittigGraphNeuralNetworks2022, singhGraphNeuralNetworks2022} to enhanced
sampling methods \cite{dietrichMachineLearningNucleation2023,
erricaDeepGraphNetwork2021}.

Here, we build upon this previously published GNN model in two ways: we apply a
hyperparameter search algorithm to further optimise the model's architecture and
improve its accuracy, and we implement an \emph{uncertainty quantification}
technique that yields confidence intervals alongside CMC predictions. This
improved model is compared against an adaptation of the group contribution
approach that determines the entire set of groups to consider using a feature
selection routine. In addition, a separate dataset is introduced so as to
perform external validation and to probe the limits of the applicability domain
of the models, as well as to analyse the uncertainty quantification. Finally, we
discuss methods to interpret both models, and demonstrate a technique that
allows one to visualise chemical space through the `eyes' of the trained GNN. We
show that this technique can help to probe the limits of the model's
applicability domain.

\section{Method}

A dataset of 202 surfactants was used, which was curated by a previous work
\cite{qinPredictingCriticalMicelle2021a} by accumulating results from several
publications. This dataset was selected because, to the authors' knowledge, it
is currently the largest public dataset of CMCs for several classes of
surfactant collected at standard conditions, in an aqueous environment between
\SIrange{20}{25}{\celsius}. These data were split into training and test
subsets, to simulate the real-world scenario of using a model to make inferences
about molecules for which no data is available. The training data was used to
fit the models, whilst test data was `locked away' until it had been decided
that the model was optimised, and the performance metrics on the test data was
used for comparison. For some models, the training data was further split into
optimisation and validation subsets; the optimisation data was used when
calculating the loss function during model fitting and the validation data was
used for on-the-fly evaluation of model performance during training.

To provide a consistent benchmark of model performance, the same train/test data
splits were used as \citet{qinPredictingCriticalMicelle2021a} and models were
also trained and evaluated using only the nonionic surfactants from the dataset,
so as to test whether generalised all-surfactant models can be as accurate as
models trained on one class of surfactant. The number of each class of
surfactant in the train and test subsets of the data are shown in
Table~\ref{tab:data-split}.

\begin{table}
    \centering
    \caption{The number of each class of surfactant contained in the train/test subsets of the CMC dataset.}
    \label{tab:data-split}
    \begin{tabular}{@{}llrrrr@{}} \toprule \multicolumn{2}{c}{Data subset} & \multicolumn{4}{c}{Number of}                                                    \\
               \cmidrule(r){1-2}\cmidrule(l){3-6}  Surfactant classes  & Train/test                    & Nonionics & Anionics & Cationics & Zwitterionics \\
               \midrule All                                            & Train                         & 110       & 30       & 31        & 9             \\
                                                                       & Test                          & 12        & 4        & 4         & 2             \\
               Nonionics                                               & Train                         & 98        &          &           &               \\
                                                                       & Test                          & 12        &          &           &               \\\bottomrule
    \end{tabular}
\end{table}

A QSPR pipeline requires choosing two essential functions: a representation
function, whose parameters are defined before training the model, and a
mathematical form that maps this representation to a prediction. The processes
by which these functions are developed are called \emph{feature engineering} and
\emph{model selection}, respectively.

\subsection{Feature engineering}

The ideal molecular representation depends on the task at hand. Ideally, it
should be compact, but complete
\cite{faberCrystalStructureRepresentations2015,himanenDScribeLibraryDescriptors2020};
`as simple as possible, but not simpler.' To that end, the representation should
contain enough information to distinguish between isomers that with distinct
properties, though concessions can be made if we restrict the model's domain and
self-impose limits on the type of isomers we expose the model to, both during
training and in use. Representations may also include descriptions of state,
such as temperature and pressure \cite{chenGraphNetworksUniversal2019}, but this
is redundant in cases where the training data spans a very limited range of
states.

In the case of the Stauff-Klevens model, the representation effectively has two
components: the category of the headgroup, and the length of the tail group. The
model technically distinguishes between isomers by imposing strict constraints
on the structure of the molecules to which it can be applied: position and
functional isomers correspond to distinctive categories, so that the model must
learn independent parameters for each of them, and the constraint on the tail
group means that chain isomers or the presence of non-alkyl groups in the carbon
chain are not permitted. Mathematically, this representation can be formalised using
a technique called \emph{one-hot encoding} and by defining the set of headgroups for which
we have data, $\{h_i \mid 0 \leq i \leq N\}$. The encoding is a vector given by

\begin{equation}
    \label{eq:one-hot}
    \vec{s}_i = \begin{cases}
        1 & \text{if headgroup is } h_i \\
        0 & \text{otherwise.}
    \end{cases}
\end{equation}

Different headgroups therefore correspond to orthogonal encodings. If we have a
set of trained parameters $\{A_i, B_i\}$ corresponding to headgroup $h_i$,
Equation~\ref{eq:klevens} can be rewritten as

\begin{equation}
    \log X_{cmc} = \mathbf{W}\vec{s} \cdot \begin{bmatrix}
        1 \\ -n_c
    \end{bmatrix},\quad \mathbf{W} = \begin{pmatrix}
        A_1 & A_2 & \dots  & A_N\\
        B_1 & B_2 & \dots & B_N\\
    \end{pmatrix}.
\end{equation}

We can try to make the approach more general by decomposing a molecule into
smaller sets of \emph{atomic environments} and representing it by the number of
each of these constituents. Because certain groups of atoms and bonds are common
in organic surfactants, the resulting feature vectors are not orthogonal, and we
can apply the model even when we have made small changes to the headgroup, or
introduce branching and other functional groups to the tail.

In this approach, the molecule is split into atomic environments up to a given
radius, $r$: each environment is centred on an atom and extends $r$ steps along
connecting bonds. Effectively, we discard the categorical encoding in favour of
introducing more continuous, count-based features, like $n_c$. The set of all
environments in the training data up to radius $r$, $\{e_i \mid 0 \leq i \leq
N\}$, is extracted and the resulting feature vector is

\begin{equation}
    \vec{c}_i = \text{Count}(e_i).
\end{equation}

Now, a change in headgroup composition is reflected in a change in subgraph
counts, and provided the new subgraph exists in our training data, the model can
adjust its prediction accordingly. Branch points in a carbon chain are
distinguished from main-chain groups, as they terminate in a \ce{CH} group,
rather than \ce{CH2}. This type of representation is called an
\emph{extended-connectivity fingerprint} (ECFP)
\cite{rogersExtendedConnectivityFingerprints2010}.

However, these fingerprints do not necessarily distinguish between all
positional isomers or chain isomers, particularly with smaller values of $r$,
nor are stereoisomers treated differently. Another potential disadvantage is
that the number of unique atomic environments is potentially very large relative
to the size of the data available, which poses a risk of overfitting.
Furthermore, larger environments necessarily envelop smaller ones, which means
that there is some duplicate information in the representation: the presence of
a \ce{(CH2)3} environment implies the presence of three \ce{CH2} environments,
so that there is multicollinearity. This redundancy can impede model fitting and
interpretation.

TODO: Graph representation description.

\subsection{Model selection}

TODO:
\begin{itemize}
    \item ECFP linear model and feature selection description.
    \item Graph neural network model description.
    \item Hyperband description.
\end{itemize}

\section{Results}

\subsection{ECFP feature selection}

The number of atomic environments remaining after each stage of the feature
selection process is reported in Table \ref{tab:ecfp-fs}. Notably, the ratio of
the number of features to the size of the training dataset is similar at
approximately \SI{74}{\%}, and so is the ratio of initial number of features to
the number of selected features, \SIrange{31}{33}{\%}. The number of features is
large compared to many of the empirical models described above, but not to the
number of graph network parameters. Furthermore, this model also aims to cover a
large part of chemical space, so a large number of parameters is to be expected.

\begin{table}
    \centering
    \caption{The number of atomic environments at each stage of the ECFP feature selection process.}
    \label{tab:ecfp-fs}
    \begin{tabular}{@{}lccc@{}} \toprule
                      & \multicolumn{3}{c}{Number of training data atomic environments}                                                            \\\cmidrule(l){2-4}
        Dataset       & Initially                                                       & Found in multiple molecules & With non-negligible weight \\\midrule
        Qin-Nonionics & 260                                                             & 201                         & 81                         \\
        Qin-All       & 410                                                             & 302                         & 134                        \\\bottomrule
    \end{tabular}
\end{table}

\subsection{Hyperband tuning}

\num{725} trials were conducted for each of the Qin training datasets. The best
hyperparameters discovered on each set are described in Table \ref{tab:hb-hps}.

\begin{table}
    \centering
    \caption{The best hyperparameters discovered during searching. The $H$
        values refer to the dimensions of the corresponding layer, see Figure
        \ref{fig:model-topology}. Values for $H_{G3}$ and $H_{D2}$ have been
        omitted where the layers weren't included in the model, and the values
        of $H_P$ were only independent for the gated attention pool, so that
        they are omitted here as well.}
    \label{tab:hb-hps}
    \begin{tabular}{@{}lcc@{}} \toprule
                        & \multicolumn{2}{c}{Best value for}            \\\cmidrule(l){2-3}
        Hyperparameter  & Qin-Nonionics                      & Qin-All  \\\midrule
        \# Graph layers & 2                                  & 3        \\
        $H_{G1}$        & 320                                & 64       \\
        $H_{G2}$        & 256                                & 64       \\
        $H_{G3}$        & --                                 & 128      \\
        Pooling layer   & Mean pool                          & Sum pool \\
        $H_P$           & --                                 & --       \\
        \# Dense layers & 2                                  & 2        \\
        $H_{D1}$        & 128                                & 256      \\
        $H_{D2}$        & --                                 & --       \\\bottomrule
    \end{tabular}
\end{table}

\subsection{Model performance}

The performance of all of the trained models on the test datasets is reported in
Table \ref{tab:evaluation}. All of the models outperformed those of the previous
work. For every task, the most accurate model was either the GNN or the combined
GNN with the GP (GNN/GP). However, the linear model's performance is surprisingly good,
considering its relative simplicity, its faster optimisation and the far smaller
number of parameters it constitutes.

\begin{table}
    \centering
    \caption{Test dataset evaluation results for the models trained in this work versus those of the previous work. The best RMSE for each task is emboldened.}
    \label{tab:evaluation}
    \begin{tabular}{@{}lccc@{}} \toprule
                                                              & \multicolumn{3}{c}{Test RMSE (\si{\log \micro M})}                                 \\\cmidrule(l){2-4}
        Model                                                 & Qin-Nonionics                                      & Qin-All       & NIST          \\\midrule
        Previous work \cite{qinPredictingCriticalMicelle2021} & 0.23                                               & 0.30          & --            \\
        ECFP                                                  & 0.19                                               & 0.26          & 1.59          \\
        GNN                                                   & \textbf{0.15}                                      & 0.29          & \textbf{1.35} \\
        GNN/GP                                                & 1.38                                               & \textbf{0.24} & 1.45          \\\bottomrule
    \end{tabular}
\end{table}

The performance on the NIST data is significantly worse than the test data
performance of every model. This suggests that the NIST data molecules are
outside of the applicability domain of the models.

Finally, the GNN/GP model's predictive performance on the Qin-Nonionics task was
very poor. This indicates that the spacing between the molecules' latent
representation vectors, determined from the corresponding GNN, was not a good
indicator of similarity with respect to CMC prediction.

\subsubsection{Uncertainty quantification}

However, the RMSE does not capture the quality of the predicted standard
deviations. One metric that captures these is the negative log likelihood (NLL)
of observing the true CMCs, given the model's predicted normal distributions:
\begin{equation}
    \text{NLL} = -\sum_n \log p_n(\hat{y}_n),
\end{equation}
where subscript $n$ is the index of the data, $\hat{y}_n$ is the true CMC value
and $p_n$ is the probability density function of the normal distribution
$\mathcal{N}(\mu_n, \sigma_n)$, where $\mu_n$ and $\sigma_n$ are the predicted
mean and standard deviation. This metric indicates the relative performance of
different models on the same data. (Note that its value scales with the size of
the data.) It does not give a good indication of the quality of any individual
model in isolation, however. The NLL values are included in the supplementary
information for comparison against future work.

To assess the models' quality individually, the predictions can be visualised
against the true CMCs in a parity plot; see Figure \ref{fig:uq-parity}.
Alternatively, a calibration plot can be used, which compares the distribution
of the residuals against the expected distribution given the models' predicted
normal distributions. The expected distribution simulates what would be observed
if the residuals were drawn from the distributions predicted by the models.
Deviations from this distribution indicate whether the model was over- or
underconfident (c.f. \citet{tranMethodsComparingUncertainty2020}). These
calibration plots are shown in Figure \ref{fig:uq-calibration}.

\begin{figure}
    \centering
    \begin{subfigure}{\textwidth}
        \includegraphics[width=\textwidth]{images/uq-parity.pdf}
        \caption{}
        \label{fig:uq-parity}
    \end{subfigure}
    \begin{subfigure}{\textwidth}
        \includegraphics[width=\textwidth]{images/uq-calibration.pdf}
        \caption{}
        \label{fig:uq-calibration}
    \end{subfigure}
    \caption{(a) Parity plots of the GNN/GP model's predicted CMCs and \SI{95}{\%}
    confidence intervals for the Qin-All and NIST datasets and (b) corresponding
    calibration plots for the test data predictions.}
\end{figure}

The S-shaped calibration curve for the Qin-All test data indicates that the
model was underconfident in its predictions: there is a spike in the number of
observed residuals that are close to the centre of the distribution. The
corresponding parity plot shows that nevertheless the predicted uncertainties
were relatively small. The NIST data calibration curve shows remarkably good
agreement with the ideal distribution, except at the top-end, which reflects the
tendency of the model to underestimate the CMCs of some of the molecules. The
relatively poor RMSE on the NIST data is somewhat ameliorated by the quality of
the uncertainty estimates.

Future efforts to improve this type of model may consider incorporating another
term in the loss function for the GNN that explicitly biases the model towards
learning a form of $\lrv{}$ that captures this similarity. Alternatively, a
variational Gaussian process could be used, which approximates the Gaussian
process using a fixed size set of `pseudo-points'
\cite{hensmanGaussianProcessesBig2013a}; this would enable the entire GNN/GP
model to be trained at once using backpropagation
\cite{moriartyUnlockNNUncertaintyQuantification2022}.


\subsection{ECFP interpretations}

TODO: find out the hybridization states of the atoms in these fingerprints
so that we can clarify the environments more:
\begin{itemize}
    \item 4201881788
    \item 1435798937
    \item 3833245231
    \item 3204830367
\end{itemize}

The weights of the ECFP models are coefficients corresponding to the scaled
counts of the selected atomic environments. Referring to Equations
\ref{eq:linear-ecfp} and \ref{eq:standard-scaling}, these coefficients indicate
the change in a predicted CMC when the count of $\mathcal{E}_m$ increases by
$s_m$ from its average, $u_m$. A more readily interpreted value can be achieved
by rescaling the coefficient, $w_m$, for an environment:
\begin{equation}
    w_m^\prime = \frac{w_m(1 - u_m)}{s_m},
\end{equation}
which indicates the difference in predicted CMC between a molecule containing
one $\mathcal{E}_m$ and a molecule without any $\mathcal{E}_m$, but which
otherwise contain exactly the same number of all the other environments. This
scaled weight can be interpreted as a rough indication of the relative
importance of different environments to determining CMC; `rough' because it may
not be physically plausible that two molecules exist that are distinguished only
by the number of $\mathcal{E}_m$ that they contain. This is particularly true of
larger environments that envelope smaller ones. The largest scaled weights for
the two ECFP models are given in Table \ref{tab:env-coefs}.

\begin{table}
    \centering
    \caption{The atomic environments with the greatest importance to CMC according to the trained ECFP models.}
    \label{tab:env-coefs}
    \begin{tabular}{@{}lSlS@{}} \toprule
        \multicolumn{2}{c}{Qin-All} & \multicolumn{2}{c}{Qin-Nonionics}                                               \\\cmidrule(r){1-2}\cmidrule(l){3-4}
        Environment                 & {Scaled weight}                   & Environment               & {Scaled Weight} \\\midrule
        \ce{(CH2)5}                 & -0.64                             & \ce{(CH2)5}               & -0.76           \\
        \ce{(CH2)3}                 & -0.55                             & \ce{(CH2)3}               & -0.69           \\
        \ce{Cl-}                    & 0.31                              & \ce{(CH2)2CH}             & -0.29           \\
        \ce{Br-}                    & 0.29                              & \ce{C}                    & -0.25           \\
        \ce{(CH2)2CH}               & -0.27                             & \ce{C(CH(OH))3CH}         & -0.19           \\
        \ce{CH2}                    & -0.23                             & \ce{CH2(CH(OH))3CH}       & 0.14            \\
        \ce{O}                      & 0.18                              & \ce{CH(CH(OH))2CH2OH}     & -0.12           \\
        \ce{OH}                     & -0.17                             & \ce{CH2CH(O)(CH(OH))2CH3} & 0.09            \\
        \ce{O(CH2)2OH}              & -0.14                             & \ce{CH3}                  & -0.06           \\
        \ce{CH2O(CH2)2OH}           & -0.14                             & \ce{CH(CH(OH))3CH}        & 0.05            \\
        \bottomrule
    \end{tabular}
\end{table}

Both models agree that alkyl chain environments constitute the top two most
important contributors to CMC, suggesting that tail length is the most important
factor. The model trained on all surfactant classes includes two counterions in
its most important environments: \ce{Cl-} and \ce{Br-}. This is to be expected;
ionic surfactants typically have much larger CMCs than nonionics, and the model
appears to distinguishes these by their counterion. The Qin-Nonionics model
identifies environments from the headgroups of sugar-based surfactants as being
important. These surfactant headgroups possessed relatively complex topologies
and therefore several environments; it may have been necessary for the model to
use many of these environments in order to accurately distinguish between their
CMCs.

TODO: Discuss NIST data and applicability domain.

\section{Discussion}

The ideal molecular
representation depends on the task at hand. Ideally, it should be compact but
complete
\cite{faberCrystalStructureRepresentations2015,himanenDScribeLibraryDescriptors2020};
`as simple as possible, but not simpler.' To that end, the representation should
contain enough information to distinguish between isomers that with distinct
properties. However, concessions can be made if we restrict the model's domain
and self-imposed limits on the type of isomers we expose the model to, both
during training and in use. Representations may also include descriptions of
state, such as temperature and pressure \cite{chenGraphNetworksUniversal2019},
but this is redundant in cases where the training data spans a very limited
range of states.

The representations employed by both the GNN and the linear models capture
topological information and the performances of all of the models on in-domain
data suggest that this is sufficient for the task of predicting CMC very
accurately. However, both of the models are unable to distinguish between
certain positional isomers, depending on the size of the atomic environments
that they consider. In the case of ECFPs, this is dictated by the radius around
each atom that is included in the fingerprint, whilst for the GNNs, this is
determined by the number of consecutive graph layers.

Increasing these parameters both increases computational cost and model
complexity, introducing more parameters and therefore requiring more data in
order to optimise them appropriately. The sensitivity analysis demonstrated that
this also increases the propensity for overfitting; however, the benchmarking
results demonstrate that using a proper selection of training samples can yield
more accurate models. In cases where there are fewer samples available, and some
chemical classes are poorly represented in the training data, the simpler,
linear model may be preferable.

One of the great advantages of using such a topological approach is that the
contributions of each molecular fragment can be explicitly determined, as shown
in the section on ECFP interpretations. In the case of the GNN, introducing a
kernel to the model, via a Gaussian process operating on the GNN's learned
latent space representations, offers a quantitative measure of molecular
similarity that can simultaneously be employed for adding uncertainty to the CMC
predictions and visualising the chemical space of the training data.
Superimposing the test data onto this graph of chemical space highlights which
molecules may have erroneous predictions, based on the fact that they are
clustered amongst training data molecules with very different chemistries.

Future efforts to improve this type of model may consider incorporating another
term in the loss function for the GNN that explicitly biases the model towards
learning a form of $\lrv{}$ that captures this similarity. Alternatively, a
variational Gaussian process could be used, which approximates the Gaussian
process using a fixed-size set of `pseudo-points'
\cite{hensmanGaussianProcessesBig2013a}; this would enable the entire GNN/GP
model to be trained at once using backpropagation
\cite{moriartyUnlockNNUncertaintyQuantification2022}.

\section{Conclusions}

Empirical models were developed and applied to predict CMCs from two datasets of
aqueous surfactants. One dataset was partitioned into training and test data
(Qin-All), and a subset of the nonionic surfactants within this data was also
used as a separate prediction task (Qin-Nonionics). The NIST dataset was
collected from a different source and contained molecules with somewhat
different chemistries than the above.

A linear model based on ECFPs demonstrated remarkably good performance,
improving on a previous work \cite{qinPredictingCriticalMicelle2021} that
applied a more complex GNN model, despite using a smaller number of parameters
and having a much faster optimization time. A new model was presented that
improved the architecture of previous work's GNN using a hyperparameter search
algorithm, which was capable of obtaining better performances than the ECFP
model on the Qin-Nonionics task and demonstrated a better ability to generalize
to the NIST dataset.

Sensitivity analysis showed that the GNN models have a tendency to overfit when
training data samples do not adequately cover the chemical space of interest.
When using small datasets, with only a few examples of certain surfactant
classes, our analysis suggests that it may be preferable to use a simpler
functional form, like the linear ECFP model.

A surrogate model was developed by feeding the latent space representation of a
molecule, learned by the GNN model, to a Gaussian process. This yielded
uncertainty estimates alongside CMC predictions. Although this model failed when
applied to the Qin-Nonionics task, it yielded the best predictive performance of
all of the models trained here for the Qin-All task, as well as providing good
uncertainty estimates on the in-domain NIST test data. This approach would allow
practitioners to gauge their confidence in the model's predictions for systems
within the applicability domain.

Finally, the kernel function that is learned while training the Gaussian process
was employed to visualize the chemical space using a molecular cartogram. By
analyzing this cartogram, it was shown that chemical intuition could be employed
to determine which molecules were likely poorly represented in the latent space,
based on the fact that they were surrounded by molecules with different
chemistries. This proves to be a useful technique for exploring the limits of a
model's applicability domain, as well as understanding why the model yields its
predictions for a given molecule based on proximity to training set molecules
within the cartogram.

This work demonstrates the potential of Gaussian processes to add uncertainty
quantification to machine learning models with minimal overhead. There is still
scope to overcome the limitations of these models with respect to small
datasets, such as Qin-Nonionics, and out-of-domain molecules, which could be
achieved by explicitly biasing the latent space.

%%%%%%%%%%%%%%%%%%%%%%%%%%%%%%%%%%%%%%%%%%%%%%%%%%%%%%%%%%%%%%%%%%%%%
%% The "Acknowledgement" section can be given in all manuscript
%% classes.  This should be given within the "acknowledgement"
%% environment, which will make the correct section or running title.
%%%%%%%%%%%%%%%%%%%%%%%%%%%%%%%%%%%%%%%%%%%%%%%%%%%%%%%%%%%%%%%%%%%%%
% \begin{acknowledgement}

% \end{acknowledgement}

%%%%%%%%%%%%%%%%%%%%%%%%%%%%%%%%%%%%%%%%%%%%%%%%%%%%%%%%%%%%%%%%%%%%%
%% The same is true for Supporting Information, which should use the
%% suppinfo environment.
%%%%%%%%%%%%%%%%%%%%%%%%%%%%%%%%%%%%%%%%%%%%%%%%%%%%%%%%%%%%%%%%%%%%%
\begin{suppinfo}

    Source code for featurisation and model training, graph neural network logs
    and metrics for hyperparameter optimisation and final training, and
    individual model predictions.

\end{suppinfo}

%%%%%%%%%%%%%%%%%%%%%%%%%%%%%%%%%%%%%%%%%%%%%%%%%%%%%%%%%%%%%%%%%%%%%
%% The appropriate \bibliography command should be placed here.
%% Notice that the class file automatically sets \bibliographystyle
%% and also names the section correctly.
%%%%%%%%%%%%%%%%%%%%%%%%%%%%%%%%%%%%%%%%%%%%%%%%%%%%%%%%%%%%%%%%%%%%%
\bibliography{camcann.bib}

\end{document}