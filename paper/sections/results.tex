\subsection{ECFP feature selection}

The number of atomic environments remaining after each stage of the feature
selection process is reported in Table \ref{tab:ecfp-fs}. Notably, the ratio of
the number of features to the size of the training dataset is similar at
approximately \SI{74}{\%}, and so is the ratio of initial number of features to
the number of selected features, \SIrange{31}{33}{\%}. The number of features is
large compared to many of the empirical models described above, but not to the
number of graph network parameters. Furthermore, this model also aims to cover a
large part of chemical space, so a large number of parameters is to be expected.

\begin{table}
    \centering
    \caption{The number of atomic environments at each stage of the ECFP feature selection process.}
    \label{tab:ecfp-fs}
    \begin{tabular}{@{}lccc@{}} \toprule
                      & \multicolumn{3}{c}{Number of training data atomic environments}                                                            \\\cmidrule(l){2-4}
        Dataset       & Initially                                                       & Found in multiple molecules & With non-negligible weight \\\midrule
        Qin-Nonionics & 260                                                             & 201                         & 81                         \\
        Qin-All       & 410                                                             & 302                         & 134                        \\\bottomrule
    \end{tabular}
\end{table}

\subsection{Hyperband tuning}

\num{725} trials were conducted for each of the Qin training datasets. The best
hyperparameters discovered on each set are described in Table \ref{tab:hb-hps}.

\begin{table}
    \centering
    \caption{The best hyperparameters discovered during searching. The $H$
        values refer to the dimensions of the corresponding layer, see Figure
        \ref{fig:model-topology}. Values for $H_{G3}$ and $H_{D2}$ have been
        omitted where the layers weren't included in the model, and the values
        of $H_P$ were only independent for the gated attention pool, so that
        they are omitted here as well.}
    \label{tab:hb-hps}
    \begin{tabular}{@{}lcc@{}} \toprule
                        & \multicolumn{2}{c}{Best value for}            \\\cmidrule(l){2-3}
        Hyperparameter  & Qin-Nonionics                      & Qin-All  \\\midrule
        \# Graph layers & 2                                  & 3        \\
        $H_{G1}$        & 320                                & 64       \\
        $H_{G2}$        & 256                                & 64       \\
        $H_{G3}$        & --                                 & 128      \\
        Pooling layer   & Mean pool                          & Sum pool \\
        $H_P$           & --                                 & --       \\
        \# Dense layers & 2                                  & 2        \\
        $H_{D1}$        & 128                                & 256      \\
        $H_{D2}$        & --                                 & --       \\\bottomrule
    \end{tabular}
\end{table}

\subsection{Model performance}

The performance of all of the trained models on the test datasets is reported in
Table \ref{tab:evaluation}. All of the models outperformed those of the previous
work. For every task, the most accurate model was either the GNN or the combined
GNN with the GP. However, the linear model's performance is surprisingly good,
considering its relative simplicity, its faster optimisation and the far smaller
number of parameters it constitutes.

\begin{table}
    \centering
    \caption{Test dataset evaluation results for the models trained in this work versus those of the previous work. The best RMSE for each task is emboldened.}
    \label{tab:evaluation}
    \begin{tabular}{@{}lccc@{}} \toprule
                                                              & \multicolumn{3}{c}{Test RMSE (\si{\log \micro M})}                                 \\\cmidrule(l){2-4}
        Model                                                 & Qin-Nonionics                                      & Qin-All       & NIST          \\\midrule
        Previous work \cite{qinPredictingCriticalMicelle2021} & 0.23                                               & 0.30          & --            \\
        ECFP                                                  & 0.19                                               & 0.26          & 1.59          \\
        GNN                                                   & \textbf{0.15}                                      & 0.29          & \textbf{1.35} \\
        GNN/GP                                                & 1.38                                               & \textbf{0.24} & 1.45          \\\bottomrule
    \end{tabular}
\end{table}

The GNN/GP model demonstrates excellent results on the Qin-All task, but a very
poor result on the Qin-Nonionics task. This poor performance was reflected in
the validation metrics, as well. This suggests that $\lrv{}$ may not indicate
the similarity of molecules' CMC values in the case of the Qin-Nonionics GNN
GNN. Future efforts to improve this type of model may consider incorporating
another term in the loss function for the GNN that biases the model towards
learning a form of $\lrv{}$ that captures this similarity. Alternatively, a
variational Gaussian process could be used, which approximates the Gaussian
process using a fixed size set of `pseudo-points'
\cite{hensmanGaussianProcessesBig2013a}; this would enable the entire GNN/GP
model to be trained at once using backpropagation
\cite{moriartyUnlockNNUncertaintyQuantification2022}.

The performance on the NIST data is significantly worse than the test data
performance of every model. This suggests that the NIST data molecules are
outside of the applicability domain of the models.

\subsubsection{ECFP interpretations}

TODO: find out the hybridization states of the atoms in these fingerprints:
\begin{itemize}
    \item 4201881788
    \item 1435798937
    \item 3833245231
    \item 3204830367
\end{itemize}

\begin{table}
    \centering
    \caption{The atomic environments with the greatest contributions to CMC according to the trained ECFP models.}
    \label{tab:env-coefs}
    \begin{tabular}{@{}lSlS@{}} \toprule
        \multicolumn{2}{c}{ECFP-All} & \multicolumn{2}{c}{ECFP-Nonionics}                                             \\\cmidrule(r){1-2}\cmidrule(l){3-4}
        Group                        & {Scaled weight}                    & Group                   & {Scaled Weight} \\\midrule
        \ce{(CH2)5}                  & -0.64                              & \ce{(CH2)5}             & -0.76           \\
        \ce{(CH2)3}                  & -0.55                              & \ce{(CH2)3}             & -0.69           \\
        \ce{Cl-}                     & 0.31                               & \ce{(CH2)2CH}            & -0.29           \\
        \ce{Br-}                     & 0.29                               & \ce{C}                  & -0.25           \\
        \ce{(CH2)2CH}                & -0.27                              & \ce{C(CH(OH))3CH}       & -0.19           \\
        \ce{CH2}                     & -0.23                              & \ce{CH2(CH(OH))3CH}     & 0.14            \\
        \ce{O}                       & 0.18                               & \ce{CH(CH(OH))2CH2OH}   & -0.12           \\
        \ce{OH}                      & -0.17                              & \ce{CH2CH(O)CH(OH)2CH3} & 0.09            \\
        \ce{O(CH2)2OH}               & -0.14                              & \ce{CH3}                & -0.06           \\
        \ce{CH2O(CH2)2OH}            & -0.14                              & \ce{CH(CH(OH))3CH}      & 0.05            \\
        \bottomrule
    \end{tabular}
\end{table}

\subsubsection{GNN interpretations}

\subsection{NIST datasets}

The layout of this should be as follows:
\begin{itemize}
    \item Describe ECFP results: the number of groups after parsing, the number found during optimisation, include best params in supplementary information.
    \item The performance of ECFP on training data.
    \item Describe hyperband results: the number of trials, the best validation error and the resulting topology.
    \item Describe performance of GNNs on test data.
    \item Describe performance of UQ on test data. Show parity plot of its predictions.
    \item Describe performance of all models on NIST data.
    \item Show parity plot of all models on NIST data.
\end{itemize}