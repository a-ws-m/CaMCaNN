Perhaps the most well established predictor for critical micelle concentration
(CMC), $X_{cmc}$, is the Stauff-Klevens relationship, first published in
\citeyear{klevensStructureAggregationDilate1953}
\cite{klevensStructureAggregationDilate1953}. It formalised the observation that
CMC decreases exponentially with an increase in the number of carbons in the
hydrocarbon tail, $n_c$:

\begin{equation}
    \label{eq:klevens}
    \log X_{cmc} = A - Bn_c, \quad B > 0
\end{equation}

where $A$ and $B$ are empirical constants that depend on the temperature and the
homologous series, i.e. the headgroup. The model is simple, yet accurate, and it
is easily interpretable: to reduce CMC, extend the surfactant's hydrocarbon
tail. Its only drawback as a predictive model is its very limited applicability
domain; each set of parameters is only applicable to surfactants with a specific
headgroup and a linear carbon tail. One of the goals of quantitative
structure-property relationship (QSPR) development is to produce models that are
general, so that we can apply them to a diverse range of compounds, design novel
molecules with target properties, and interpret the models' results to glean
chemical insights.

To that end, there have been a wealth of investigations into making more general
models for CMC prediction, which are discussed in Section \ref{sec:background}.
The two fundamental differences between them are the choice of molecular
descriptors, which are the numerical features that form the basis set for the
model inputs, and the functional form of the approximator that maps the
descriptor to the property prediction. 

Recently, an approach based on graph neural networks (GNNs) has produced highly
accurate predictions, whilst having the advantage of being applicable to
nonionic, cationic, anionic and zwitterionic surfactants simultaneously
\cite{qinPredictingCriticalMicelle2021}. Such neural networks have many
trainable parameters and a complex functional form. This ensures their
versatility as general approximators, but makes them highly susceptible to
overfitting (CITATION). By using such complex models, we also abandon the
parsimony exhibited by Stauff-Klevens, and chemical insights can be much more
difficult to derive. Furthermore, deep neural networks' ostensible
`universality' can be misleading: extrapolating the model's results to
out-of-domain molecules (ones that are `dissimilar' from the training data) will
yield unreliable and potentially misleading predictions.

In this article, we assess a spectrum of empirical models for CMC prediction,
with the intent of finding the optimum trade-off between accuracy, generality
and interpretability for this task, by correlating our models' interpretations
with published research. We also apply a technique for adding uncertainty
quantification to GNNs, which can indicate whether a molecule is within the
model's applicability domain and therefore whether a given prediction is
reliable. We aim to demonstrate a best-practices approach to applying machine
learning to QSPR tasks on small datasets.
