The critical micelle concentration (CMC) of a surfactant defines the concentration above which the surfactant monomers self-assemble in solution to form micelles. It is an important property because the formation of micelles affects many interfacial phenomena \cite{rosenSurfactantsInterfacialPhenomena2012} micelles can encapsulate hydrophobic molecules to form useful complexes, but the formation of micelles also inhibits certain processes, like the aggregation of surfactants at interfaces in order to reduce surface tension.

Perhaps the most well-established predictor for CMC, $X_{cmc}$, is the Stauff-Klevens relationship, first published in \citeyear{klevensStructureAggregationDilate1953} \cite{klevensStructureAggregationDilate1953}. It formalised the observation that CMC decreases exponentially with an increase in the number of carbons in the hydrocarbon tail, $n_c$:

\begin{equation}
    \label{eq:klevens}
    \log X_{cmc} = A - Bn_c, \quad B > 0
\end{equation}

where $A$ and $B$ are empirical constants that depend on the temperature and the homologous series, i.e. the headgroup. The model is simple yet accurate, and it is easily interpretable: to reduce CMC, it is sufficient to extend the surfactant's hydrocarbon tail thus defining an easy-to-apply qualitative heuristic. Its drawback as a predictive model is its very limited applicability domain; each set of parameters is only applicable to surfactants with a specific headgroup and a linear carbon tail.
One of the goals of the quantitative structure-property relationship (QSPR) development is to produce models that are general so that we can apply them to a diverse range of compounds, design novel molecules with target properties, and
interpret the models' results to glean chemical insights.

There have been a wealth of investigations into making more general models for
CMC prediction; here, a brief review of some diverse and promising approaches
for predicting CMCs of aqueous, single-surfactant systems will be given.

Broadly, CMC predictive models take four forms: empirical, semi-empirical,
theoretical and simulated, or some combination of these. Here we will focus on
predictive models for aqueous solutions containing a single surfactant and
discuss some of the trade-offs between the different approaches with regards to
speed, universality and interpretability.

Theoretical approaches have the potential to be the most useful type of
predictive model, if they are accurate and applicable to the desired system, as
they are directly related to scientific knowledge and their results can be
understood in terms of well studied principles.
\citet{puvvadaMolecularThermodynamicApproach1990} derived a phenomenological
model for studying aqueous nonionic surfactant systems that enabled CMC
prediction, as well as modelling of other properties across a range of
temperatures. The model they developed was the product of decomposing the
process of micellisation into discrete steps that they could describe
thermodynamically so as to yield an description of the free energy of
micellisation in terms of a set of molecular parameters:

\begin{itemize}
    \item The tail length, described as the number of carbon atoms.
    \item The average cross-sectional area of the headgroup, which controls the
          steric contribution to the free energy. This must be estimated.
    \item The Tolman length of the tail, which effectively describes the
          thickness of an `interaction region' around the tail
          \cite{demiguelGibbsThermodynamicsSurface2021}. This must also be estimated.
\end{itemize}

A functional form to estimate the parameters was described for linear, nonionic,
polyoxyethylene alcohol surfactants. The model attained impressive accuracy for
some predictions: a root-mean-squared error (RMSE) of approximately
\SI{0.14}{\log \micro M} for the group \ce{C10E_i}, where $i \in [3, 6]$, and
\SI{0.21}{\log \micro M} for the group \ce{C12E_j}, where $j \in [3, 8]$.
However, for other systems, like \ce{C8E6}, the error is much larger. The
authors expect that this inaccuracy is because the model overestimates the CMC
values for systems in which the micelles do not grow.

The connection that this model established between a small set of physically
meaningful properties that can be estimated and emergent properties of
surfactants is extremely useful, especially because it does not explicitly
require fitting to any experimental data. However, the procedures described for
estimating the Tolman length are only applicable for linear hydrocarbon chains;
not the branched case, or for heterogeneous tail groups. Estimating the average
cross-sectional area of the head group may also not be trivial.

Semi-empirical approaches are grounded in theory, but have parameters that are
optimised based on experimental data. Many semi-empirical approaches to CMC
prediction can be described as \emph{segment-based} methods, whereby the
surfactant is decomposed into discrete segments which correspond to groups of
atoms and bonds.

\citet{liStudiesUNIQUACSAFT1998} applied a segment-based UNIQUAC model
(s-UNIQUAC) and a SAFT equation of state to predict CMCs of linear
polyoxyethylene alcohols, by first deriving expressions for the activity
coefficient of a surfactant in water. In the s-UNIQUAC model, a segment-based
local-composition model was used and the fugacity could then be approximated
using the fitted interaction energies between the segments and water. In this
case, the segments used were \ce{C2H4} and \ce{C2H4O}. In the SAFT approach, the
surfactant was treated as a chain of soft-sphere segments in order to first
derive the Helmholtz energy of the solution and from that to derive the
fugacity. In this case, the segments used were \ce{CH2}/\ce{CH3} (these were
treated as the same segment) and \ce{C2H4O}. The interaction energies of the
segments were fitted, as well as parameters of a function describing the soft
sphere diameter of a segment in a chain in terms of the chain length.

\citet{chengCorrelationCriticalMicelle2005} compared the performance of these
models on a larger dataset, alongside three other models. Two of these were
segment-based models: the polymer-NRTL model \cite{liStudiesUNIQUACSAFT1998} and
a UNIFAC model \cite{voutsasPredictionCriticalMicelle2001}, both of which were
cited as inspirations for the s-UNIQUAC model. The authors also employed their
own modified Aranovich and Donohue (m-AD) model. The m-AD model calculates the
CMC as a mole fraction, $x_S^L$, approximating it as the reciprocal of the
limiting value of the surfactant's activity coefficient in aqueous solution,
$\gamma_S^{L,\infty}$:

\begin{equation}
    \label{eq:m-AD}
    x_S^L = \frac{1}{\gamma_S^{L,\infty}}
\end{equation}

The m-AD model considers the exchange equilibrium on a three-dimensional lattice
of infinitely separated solvent and solute molecules in order to determine
$\gamma_S^{L,\infty}$. Notably, the m-AD model is not a segment-based model.
Instead, the authors fitted an interchange energy, $\Delta$, separately for each
molecule. Of course, if a new parameter must be fitted for every molecule, a
model has no predictive ability. Therefore, correlations were examined between
$\Delta$ and other, readily calculated surfactant properties, which will be
discussed later.

Where data from literature was available, the predictive performance of the
models on the molecular series \ce{C_nE6}, \ce{C_nE8}, \ce{C_nE9}, \ce{C10E_n}
and \ce{C12E_n} were compared, and the resulting RMSEs are summarised in Table
\ref{tab:segment-methods}.

\begin{table}
    \caption{Comparison of the RMSEs of selected models on polyoxyethylene
        alcohols. Data from \citet{chengCorrelationCriticalMicelle2005}.}
    \label{tab:segment-methods}
    \begin{tabular}{lr}
        \toprule
        \multicolumn{1}{c}{Model} & \multicolumn{1}{c}{RMSE (\si{\log \micro M})} \\\midrule
        p-NRTL                    & 0.18                                          \\
        s-UNIQUAC                 & 0.14                                          \\
        SAFT                      & \textbf{0.06}                                 \\
        UNIFAC                    & 0.14                                          \\
        m-AD                      & 0.11                                          \\\bottomrule
    \end{tabular}
\end{table}

TODO: Talk about geometric, topological and electronic descriptors, as well as
simulation approaches.

The two fundamental differences between them are the choice of molecular
descriptors, the numerical features that form the basis set for the model
inputs, and the functional form of the approximator that maps the descriptor to
the property prediction. 

Recently, an approach based on graph neural networks (GNNs) has produced highly accurate predictions whilst being applicable to nonionic, cationic, anionic and zwitterionic surfactants simultaneously
\cite{qinPredictingCriticalMicelle2021}. Neural networks have many trainable parameters and a complex functional form. This ensures their versatility as universal approximators but makes them highly susceptible to overfitting \cite{bejaniSystematicReviewOverfitting2021}. Using such complex models, we also abandon the parsimony exhibited by Stauff-Klevens, and chemical insights can be much more difficult to derive. Furthermore, deep neural networks' ostensible `universality' can be misleading: extrapolating the model's results to out-of-domain molecules (ones that are `dissimilar' from the training data) will yield unreliable and potentially misleading predictions.

In this article, we develop two families of models of very different complexity: a linear model and a GNN. We evaluate the difference in performance and interpretability of the models. We also apply a technique for adding uncertainty
quantification to the GNN, which can indicate whether a molecule is within the model's applicability domain and, therefore, whether a given prediction is reliable.
