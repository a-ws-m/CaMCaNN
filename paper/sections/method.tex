A dataset of 202 surfactants was used, which was curated by a previous work
\cite{qinPredictingCriticalMicelle2021a} by accumulating results from several
publications. This dataset was selected because, to the authors' knowledge, it
is currently the largest public dataset of CMCs for several classes of
surfactant collected at standard conditions, in an aqueous environment between
\SIrange{20}{25}{\celsius}. These data were split into training and test
subsets, to simulate the real-world scenario of using a model to make inferences
about molecules for which no data is available. The training data was used to
fit the models, whilst test data was `locked away' until it had been decided
that the model was optimised, and the performance metrics on the test data was
used for comparison. For some models, the training data was further split into
optimisation and validation subsets; the optimisation data was used when
calculating the loss function during model fitting and the validation data was
used for on-the-fly evaluation of model performance during training.

To provide a consistent benchmark of model performance, the same train/test data
splits were used as \citet{qinPredictingCriticalMicelle2021a} and models were
also trained and evaluated using only the nonionic surfactants from the dataset,
so as to test whether generalised all-surfactant models can be as accurate as
models trained on one class of surfactant. The number of each class of
surfactant in the train and test subsets of the data are shown in
Table~\ref{tab:data-split}.

\begin{table}
    \centering
    \caption{The number of each class of surfactant contained in the train/test subsets of the CMC dataset.}
    \label{tab:data-split}
    \begin{tabular}{@{}llrrrr@{}} \toprule \multicolumn{2}{c}{Data subset} & \multicolumn{4}{c}{Number of}                                                    \\
               \cmidrule(r){1-2}\cmidrule(l){3-6}  Surfactant classes  & Train/test                    & Nonionics & Anionics & Cationics & Zwitterionics \\
               \midrule All                                            & Train                         & 110       & 30       & 31        & 9             \\
                                                                       & Test                          & 12        & 4        & 4         & 2             \\
               Nonionics                                               & Train                         & 98        &          &           &               \\
                                                                       & Test                          & 12        &          &           &               \\\bottomrule
    \end{tabular}
\end{table}

A QSPR pipeline requires choosing two essential functions: a representation
function, whose parameters are defined before training the model, and a
mathematical form that maps this representation to a prediction. The processes
by which these functions are developed are called \emph{feature engineering} and
\emph{model selection}, respectively.

\subsection{Feature engineering}

The ideal molecular representation depends on the task at hand. Ideally, it
should be compact, but complete: ``as simple as possible, but not simpler.''

We can generalise further and skip the classification by applying a similar
linear model to the headgroup, based on the number of and type of chemical
groups that constitute it. In this approach, we split the molecule into
\emph{atomic environments} up to a certain radius, $r$: each subgraph is centred
on an atom and extends $r$ steps along connecting bonds. Now, a small change in
headgroup composition is reflected in a change in subgraph counts, and provided
the new subgraph exists in our training data, the model can adjust its
prediction accordingly. Tail group branching is also considered, to an extent:
branch points in a carbon chain are distinguished from the rest, as they
constitute a \ce{CH} subgraph, rather than a \ce{CH2}.

Still, this approach relies on a discrete representation of the surfactant and,
to ensure a good fit, all subgraphs of interest must be well represented in the
dataset. Neural network approaches that work on molecular graphs are capable of
learning a continuous representation of surfactants. They have been explored
previously for CMC prediction and demonstrated promising results. However,
although the continuous representation ensures that inferences can be made even
for molecules with chemistry that is completely different from the training
data, neural networks cannot reliably extrapolate in this way. In the case of
the ECFP model, the discretisation gives a clear indicator to the researcher as
to whether the model can be reliably applied: if a molecule contains subgraphs
that do not appear in the training data, its prediction is less reliable. It is
difficult to assess the same for a neural network.