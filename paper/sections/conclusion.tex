Empirical models were applied to predict CMCs from two datasets. One dataset was
partitioned into training and test data (Qin-All), and a subset of the nonionic
surfactants within this data were also used as a separate prediction task
(Qin-Nonionics). The NIST dataset was collected from a different source and
contained some molecules with very different chemistries than the above.

A linear model based on ECFPs demonstrated remarkably good performance,
improving on a previous work \cite{qinPredictingCriticalMicelle2021} that
applied a more complex GNN model, despite using a smaller number of parameters
and having a much faster optimisation time. A new model was presented that
improved the architecture of previous work's GNN and was capable of obtaining a
better performance than the ECFP model on the Qin-Nonionics task and
demonstrated a better ability to generalise to the NIST dataset.

Finally, a surrogate model was developed by feeding the latent space
representation of a molecule, learned by the GNN model, to a Gaussian process.
This yielded uncertainty estimates alongside CMC predictions. Although this
model appeared to fail when applied to the Qin-Nonionics task, it yielded the
best predictive performance of all of the models for the Qin-All task as well as
providing a good quality of uncertainty estimates, which allow researchers to
gauge their confidence in the model's predicitons.

TODO: Write about applicability domain.