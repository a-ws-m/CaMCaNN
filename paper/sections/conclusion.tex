Empirical models were developed and applied to predict CMCs from two datasets of
aqueous surfactants. One dataset was partitioned into training and test data
(Qin-All), and a subset of the nonionic surfactants within this data was also
used as a separate prediction task (Qin-Nonionics). The Complementary dataset was
collected from a different source and contained molecules with somewhat
different chemistries than the above.

A linear model based on ECFPs demonstrated remarkably good performance,
improving on a previous work \cite{qinPredictingCriticalMicelle2021} that
applied a more complex GNN model, despite using a smaller number of parameters
and having a much faster optimization time. A new model was presented that
improved the architecture of previous work's GNN using a hyperparameter search
algorithm, which was capable of obtaining better performances than the ECFP
model on the Qin-Nonionics task and demonstrated a better ability to generalize
to the Complementary dataset.

Sensitivity analysis showed that our GNN models have a tendency to overfit when
training data samples do not adequately cover the chemical space of interest.
When using small datasets, with only a few examples of certain surfactant
classes, our analysis suggests that it may be preferable to use a simpler
functional form, like the linear ECFP model.

A surrogate model was developed by feeding the latent space representation of a
molecule, learned by the GNN model, to a Gaussian process. This yielded
uncertainty estimates alongside CMC predictions. Although this model failed when
applied to the Qin-Nonionics task, it yielded the best predictive performance of
all of the models trained here for the Qin-All task, as well as providing good
uncertainty estimates on the in-domain Complementary test data. This approach would allow
practitioners to gauge their confidence in the model's predictions for systems
within the applicability domain.

Finally, the kernel function learned while training the Gaussian process was
employed to visualize the chemical space using a molecular cartogram. By
analyzing this cartogram, it was shown that chemical intuition could be employed
to determine which molecules were likely poorly represented in the latent space,
based on the fact that they were surrounded by molecules with different
chemistries. This promises to be a useful technique for exploring the limits of
a model's applicability domain, as well as understanding why the model yields
its predictions for a given molecule based on proximity to training set
molecules within the cartogram.

This work demonstrates the potential of Gaussian processes to add uncertainty
quantification to machine learning models with minimal overhead. There is still
scope to overcome the limitations of these models with respect to small
datasets, such as Qin-Nonionics, and out-of-domain molecules, which could be
achieved by explicitly biasing the latent space.