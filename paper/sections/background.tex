Theoretical approaches have the potential to be the most useful type of
predictive model if they are accurate and applicable to the desired system, as they are directly related to scientific knowledge, and their results can be understood in terms of well-studied principles.
\citet{puvvadaMolecularThermodynamicApproach1990} derived a phenomenological model for studying aqueous nonionic surfactant systems that enabled CMC prediction and modelling other properties across a range of temperatures. The model they developed was the product of decomposing the process of micellisation into discrete steps that they could describe thermodynamically so as to yield a description of the free energy of micellisation in terms of a set of molecular parameters:

\begin{itemize}
    \item The tail length, defined as the number of carbon atoms.
    \item The average cross-sectional area of the headgroup, which controls the
          steric contribution to the free energy. This must be estimated.
    \item The Tolman length of the tail, which effectively describes the
          thickness of an `interaction region' around the tail
          \cite{demiguelGibbsThermodynamicsSurface2021}. This must also be estimated.
\end{itemize}

A functional form to estimate the parameters was described for linear, nonionic, polyoxyethylene alcohol surfactants. The model attained impressive accuracy for some predictions: a root-mean-squared error (RMSE) of approximately \SI{0.14}{\log \micro M} for the group \ce{C10E_i}, where $i \in [3, 6]$, and \SI{0.21}{\log \micro M} for the group \ce{C12E_j}, where $j \in [3, 8]$.
However, the error is much larger for other systems, like \ce{C8E6}. The
authors expect that this inaccuracy is because the model overestimates the CMC values for systems in which the micelles do not grow.

The connection this model established between a small set of physically meaningful properties that can be estimated and emergent properties of surfactants is extremely useful, especially because it does not explicitly require fitting to any experimental data. However, the procedures described for estimating the Tolman length are only applicable to linear hydrocarbon chains, not branched nor heterogeneous tail groups. Estimating the average cross-sectional area of the head group may also not be trivial.

Greater generalisability and better accuracy can be realised by considering the
process of aggregation on a more precise level: identifying the behaviour of
individual atoms, or small groups thereof. Simulation approaches can model the
interaction of these units with each other and derive the potential energy of a
configuration from these calculations. By combining this information with
entropy considerations, statistical mechanics can indicate free energy changes,
such as those associated with micellisation, from which the CMC can be derived.

Molecular dynamics (MD) simulations treat individual atoms, in the all-atomistic
(AA) approach, or groups of atoms, in the coarse-grained (CG) approach, as
particles in a box that exert force on each other. This allows the particles'
movement to be simulated by iteratively updating their positions and momenta in
discrete `time steps'. \citet{jorgeMolecularDynamicsSimulation2008} used two
types of AA approach to simulate the self-assembly of
\textit{n}-decyltrimethylammonium bromide: one approach that explicitly
considered hydrogens, and another, united-atom approach that grouped hydrogens
with their respective connecting atom. They then estimated the CMC by
considering the concentration of `free' surfactants, i.e. surfactants that were
not in micelles, which they defined as an aggregate containing five or more
surfactants. The more accurate model for determining CMC, the united-atom
approach, yielded a CMC with an error of \SI{0.23}{\log \micro M}. The author
asserted that the simulation size, i.e. the number of particles, was too small
to obtain a more accurate value of CMC. This highlights an issue with using MD
to predict CMC: careful consideration must be given to ensure that the system is
large enough and that enough timesteps have been performed to ensure
convergence, but increasing these parameters leads to an increase in
computational cost.

The computational cost in increasing the system size can be mitigated using
coarse-graining, which reduces the number of interactions that must be computed
by grouping atoms into beads, as well as enabling larger time steps, meaning
that the same timescale can be simulated with fewer rounds of calculation
\cite{fitzgeraldMultiscaleModelingNanomaterials2015}. These beads must be
modelled differently than individual atoms and there are a few approaches to
this. \citet{vishnyakovPredictionCriticalMicelle2013} used dissipative particle
dynamics (DPD) to model the CMCs of \ce{C8E8}, dodecyldimethylamineoxide (DDAO)
and \textit{N}-decanoyl-\textit{N}-methyl-\textsc{D}-glucamide (MEGA-10). They
proposed a methodology for obtaining reasonable parameters to describe the bead
interactions based on known infinite dilution coefficients, $\gamma_\infty$.
Their calculated CMC for \ce{C8E8} had an error relative to the average
experimental value equal to \SI{0.07}{\log \micro M}, for DDAO it was
\SI{0.06}{\log \micro M}, and for MEGA-10 it was \SI{0.06}{\log \micro M}.

Clearly, this approach can produce extremely accurate results. However, it is
limited to compounds that can be parameterised by a known $\gamma_\infty$.
Although this value can also be simulated, this must be done very carefully as
the error in this value will be compounded. Furthermore, the authors do not
discuss parameterising ionic surfactants, for which more complex interactions
must be considered. Despite the reduced cost relative to AA-MD models of the
same system scales, CG-MD approaches are still one of the more expensive
approaches to CMC prediction. But MD simulations give very deep insight into the
processes occurring at a molecular level; its results yield a precise
description of the arrangement of molecules in a system, their shape and
aggregation number, which is a level of detail other approaches do not come
close to achieving.

Another approach is to try to reduce the scale of the simulation by only
considering the effect of individual molecules and local interactions. The
conductor-like screening model (COSMO) provides a tool for decomposing the
problem in this way, by treating a molecule as a cavity with a charged surface
in a solvent that acts as a dielectric continuum
\cite{klamtCOSMONewApproach1993}. The cavity's surface is described by the
solvent-accessible surface of the molecule. The geometry of this surface
combined with a segment-wise description of its polarising charges is called the
COSMO-surface and it can be calculated using density functional theory (DFT).

COSMO for realistic solvation (COSMO-RS) adapts the model for more complex types
of solvent \cite{klamtCOSMORSAlternativeSimulation2010}. Solvents only act like
a dielectric continuum when they are capable of perfectly screening the
COSMO-surface of a solute, i.e. every point on the molecule's COSMO surface is
matched by a point of opposite polarity due to the configuration of the solvent
molecules. Due to entropy considerations, as well as the fact that some solvent
molecules do not have appropriate polar charges to match-up in this way, this is
often a poor approximation. COSMO-RS uses statistical mechanics to determine the
probability distribution describing how surface charge densities align between
two molecules. This allows chemical potentials to be determined.

COSMO-RS has two major advantages over MD simulations. Firstly, it removes the
need to study the configuration space of an entire ensemble of molecules; the
problem is reduced just to local interactions. This means the computational cost
does not scale with respect to the system size, but also that it can only
consider homogeneous, unstructured systems. Another advantage is that it
naturally considers effects from dipole and quadrupole moments, which can be
hard to capture using the classical forces that drive MD
\cite{klamtCOSMORSAlternativeSimulation2010}.

\citet{turchiFirstprinciplesPredictionCritical2022} used COSMO-RS to predict
CMCs by treating a micelle as a separate phase and then considering the
two-phase equilibrium between the micelle and an aqueous phase containing free
surfactants. They modelled the micellar `phase' using two strategies. Their
first strategy was to treat the micellar phase as being equivalent to a bulk,
homogeneous phase of surfactant. The CMC could then be determined by the
equilibrium surfactant concentration in the aqueous phase. In actuality,
micelles are structured and inhomogeneous; the authors argued that this
approximation is more valid as the difference in polarity between the head and
tail of a surfactant is reduced, which was the case for the majority of nonionic
surfactants they considered.

Their second strategy was to consider the micellar phase as a bulk, homogeneous
phase of an oil, whose chemistry was analogous to that of the surfactant's tail
group. They then implemented an iterative procedure to calculate the interfacial
tension (IFT) between the oil and aqueous surfactant phases at different
concentrations of surfactant. The concentration at which the IFT is zero yields
the CMC prediction. The premise of this approach is that the oil phase is
representative of a micelle's interior, containing primarily tail groups, which
is particularly true as the interactions between head and tail groups become
more unfavourable. This is the case for highly polar head groups: primarily for
ionic surfactants.

The authors recommended applying both strategies and using the lower result as
the CMC prediction. With the combined strategies, they attained an RMSE of
\SI{0.81}{\log \micro M} on a dataset of 24 surfactants, containing a mix of
ionic, nonionic and zwitterionic. Excluding the two worst predictions gives a
much more favourable RMSE of \SI{0.55}{\log \micro M}. These results are
relatively inaccurate, but it is impressive that the technique can be applied
across all classes of surfactant, whilst requiring significantly lower
computational cost than MD. Furthermore, the authors were able to determine
which of the nonionic surfactants had a low propensity to form micelles by
whether the predicted IFT between water and surfactant phases was large at
equilibrium, which is a testament to the model's interpretability.

Other approaches have extended COSMO-RS to explicitly account for the internal
structure of micelles, such as COSMOmic, which treats a micelle as being made of
concentric layers that each have their own surface charge profiles
\cite{klamtCOSMOmicMechanisticApproach2008}. To compute these charge profiles,
the layer's composition with respect to individual atoms of the surfactant must
be known; this can be determined using MD. The combination of layer-wise atomic
distributions and the COSMO-surface associated with each atom gives the layer's
surface charge density profile. COSMOmic considers how the COSMO-surface of a
surfactant would intersect with the layers' surface charge profiles when it is
randomly positioned and oriented within the micelle, in order to determine the
partition coefficient of inserting a surfactant molecule into the micelle.

\citet{jakobtorweihenPredictingCriticalMicelle2017} calculated CMCs using
COSMOmic by first performing MD simulations to attain the layer-wise atomic
distributions. As discussed above, MD by itself can calculate the CMC of a
system, which might suggest that the COSMOmic step is redundant. However, one
can expect the layer-wise atomic distributions to converge much more rapidly
than equilibrium concentrations if the micelle is \emph{pre-assembled}, meaning
that the initial configuration for the simulation constitutes a guess for the
micelle's structure. The authors then predicted the CMCs of several
polyoxyethylene alcohols by determining the partition coefficients of inserting
the respective surfactant monomer into a micelle.

Another consideration with this technique is that molecules may adopt several
conformations, each with a unique COSMO-surface. Although it was found that the
choice of conformer used to describe the micellar layer-wise charge density
profiles has a negligible effect on the results
\cite{jakobtorweihenCombinationCOSMOmicMolecular2013}, the conformer used for
the partitioning surfactant is important. With the best choice of conformer, the
authors achieved an RMSE of $\sim \SI{0.36}{\log \micro M}$ on predictions for
\ce{C_$i$E6} surfactants, where $i \in \{6, 8, 10, 12, 14, 16\}$. However, when
considering surfactants with fixed tail length but varying head group size,
\ce{C10E_$j$}, $j \in {4, 6, 8}$, the authors could not identify a consistent
conformer selection that would not yield significant outliers.

The technique is a promising compromise between the accuracy of the full-MD
approach and the lower computational cost of COSMO-RS. However, to improve the
reliability, a method for accurately identifying a good choice of conformer and
the shape of the pre-assembled micelle must be realised.

Finally, COSMOplex is a recent extension of COSMOmic that removes the need to
perform and initial MD simulation to determine the micellar structure
\cite{klamtCOSMOplexSelfconsistentSimulation2019}. Instead, it optimises a guess
of the initial structure using a self-consistent approach by considering
COSMOmic's predicted probability distribution of the partitioning surfactant's
configuration within the micelle to iteratively yield new estimates for the
layer-wise charge distributions. In this way, the authors predicted the CMCs of
10 nonionic surfactants with varied head and tail group chemistries, achieving
an RMSE of \SI{0.86}{\log \micro M}. The authors note that these results are
preliminary and stand to be improved.

The advantage of this technique is both a decrease in computational cost and
improved parsimony; by eliminating the MD step, there is reduced complexity and
the aforementioned difficulties with force field considerations are irrelevant.
One of the difficulties with the technique is its sensitivity to the original
guess for the atomic distributions
\cite{klamtCOSMOplexSelfconsistentSimulation2019}. Improving the guess or the
optimisation technique to be less sensitive to local minima would therefore lead
to more accurate predictions.

COSMO techniques are an interesting method to reduce the dimensionality of
self-assembly structure studies whilst maintaining a solid grounding in theory,
which allows a researcher to directly relate the results to scientific
principles whilst understanding the limits of the approximations \emph{a
priori}. However, \citet{herbertDielectricContinuumMethods2021} have argued that
the discussion around COSMO-RS indicates that there is not enough detail in the
literature to allow for an open source implementation, meaning that many modern
extensions are only available in the proprietary software package
\textsc{COSMOTHERM} \cite{eckertFastSolventScreening2002}. The closed-source
nature of the technology is a barrier to scientific scrutiny and, crucially,
improvement of the model.

Another approach is to use an equation of state with a much less complex
functional form to determine the CMC. The reduction in complexity can be
achieved using a semi-empirical method that is parameterised by fitting to
experimental data.

\citet{liStudiesUNIQUACSAFT1998} applied a segment-based UNIQUAC model
(s-UNIQUAC) and a SAFT equation of state to predict CMCs of linear
polyoxyethylene alcohols by first deriving expressions for the activity
coefficient of a surfactant in water. In the s-UNIQUAC model, a segment-based
local-composition model was used, and the fugacity could then be approximated
using the fitted interaction energies between the segments and water. In this
case, the segments used were \ce{C2H4} and \ce{C2H4O}. In the SAFT approach, the
surfactant was treated as a chain of soft-sphere segments in order to first
derive the Helmholtz energy of the solution and, from that, derive the fugacity.
In this case, the segments used were \ce{CH2}/\ce{CH3} (these were treated as
the same segment) and \ce{C2H4O}. The interaction energies of the segments were
fitted, as well as parameters of a function describing the soft sphere diameter
of a segment in a chain in terms of the chain length.

\citet{chengCorrelationCriticalMicelle2005} compared the performance of these
models on a larger dataset alongside three other models. Two of these were
segment-based models: the polymer-NRTL model \cite{liStudiesUNIQUACSAFT1998} and
a UNIFAC model \cite{voutsasPredictionCriticalMicelle2001}, both of which were
cited as inspirations for the s-UNIQUAC model. The authors also employed their
own modified Aranovich and Donohue (m-AD) model. The m-AD model calculates the
CMC as a mole fraction, $x_S^L$, approximating it as the reciprocal of the
limiting value of the surfactant's activity coefficient in an aqueous solution,
$\gamma_S^{L,\infty}$:

\begin{equation}
    \label{eq:m-AD}
    x_S^L = \frac{1}{\gamma_S^{L,\infty}}
\end{equation}

The m-AD model considers the exchange equilibrium on a three-dimensional lattice
of infinitely separated solvent and solute molecules in order to determine
$\gamma_S^{L,\infty}$. Notably, the m-AD model is not a segment-based model;
instead, the authors fitted an interchange energy, $\Delta$, separately for each
molecule. Of course, if a new parameter must be fitted for every molecule, a
model has no predictive ability. Therefore, correlations were examined between
$\Delta$ and other, readily calculated surfactant values: the Kier-Hall
zero-order index (KH0) of the tail groups, which indicates  and the total
molecular energy of the surfactant.

Where data from the literature was available, the predictive performance of the
models on the molecular series \ce{C_nE6}, \ce{C_nE8}, \ce{C_nE9}, \ce{C10E_n}
and \ce{C12E_n} were compared, and the resulting RMSEs are summarised in Table
\ref{tab:segment-methods}. The models all have a reasonably good accuracy, but
the SAFT model in particular is excellent.

\begin{table}
    \caption{Comparison of the RMSEs of selected models on polyoxyethylene
        alcohols. Data from \citet{chengCorrelationCriticalMicelle2005}.}
    \label{tab:segment-methods}
    \begin{tabular}{lr}
        \toprule
        \multicolumn{1}{c}{Model} & \multicolumn{1}{c}{RMSE (\si{\log \micro M})} \\\midrule
        p-NRTL                    & 0.18                                          \\
        s-UNIQUAC                 & 0.14                                          \\
        SAFT                      & \textbf{0.06}                                 \\
        UNIFAC                    & 0.14                                          \\
        m-AD                      & 0.11                                          \\\bottomrule
    \end{tabular}
\end{table}

Segment-based semi-empirical methods are very promising for predicting CMCs
within a class of surfactants. Their major drawback is that they are only
applicable to molecules that can be decomposed into segments that have trained
parameters. In addition, they must respect the limitations of the theories they
are based upon. For example, it was described earlier how the SAFT model needs
special treatment to be applied to these polyoxyethylene alcohol surfactants,
requiring a function to describe a segment's soft sphere diameter with respect
to the chain length. A more complex function would be required when branching is
introduced, or when chains can be made up of combinations of different units,
and it is difficult to make these adaptations.

Finally, purely empirical methods have a very heavy reliance on data abundance.
Empirical methods offer a way of making predictions even when a unified theory
that ties in the behaviour of several classes of molecules is not readily
apparent, or else when the theory requires computationally expensive procedures
to put into practice. However, without an underlying theory, the assumptions and
therefore the limitations of the model are not well defined and it is possible
for the model to `learn' trends that contradict established scientific
intuition.

Empirical QSPR methods require validation to determine their reliability and
applicability domain
\cite{veerasamyValidationQSARModelsstrategies2011,tropshaBestPracticesQSAR2010,leonardSelectionTrainingTest2006},
which is often achieved by partitioning the available data into \emph{training}
and \emph{test} subsets; the former is used for optimising the model's
parameters, but the latter is `hidden' from the model until training is
complete, and the prediction metrics on the test data indicate how the model can
be expected to perform in general. The test set should span the chemical space
in which the model is intended to be applied
\cite{leonardSelectionTrainingTest2006}. At a high level, this means that all
classes of surfactant that are `advertised' as being within the applicability
domain of the model should be represented in both the test and train data
subsets.

Every empirical QSPR method is characterised by two features: the choice of
molecular descriptors, which are the numerical features that make up the model
inputs, and the functional form of the approximator that maps the descriptors to
the property prediction. This functional form includes the parameters that are
optimised based upon the training data.

\citet{matteiModelingCriticalMicelle2013} extended the Marrero and Gani
group-contribution method \cite{ganiAutomaticCreationMissing2005} to predict the
CMCs of 161 nonionic surfactants. This is another segment-based method, but here
the pretence of describing the problem in terms of the effect of interacting
segments is abandoned. Instead, each segment is treated as an independent
\emph{group} that contributes to the prediction in an additive, linear fashion.

The descriptors in this case are the number of each group present in a molecule,
which are identified using the constraint that some groups are not allowed to
overlap with one another to avoid redundant information in the descriptor. In
the original method \cite{ganiAutomaticCreationMissing2005}, different `orders'
of groups were identified; the first-order groups are forbidden from overlapping
with one another and they are formulated so that any molecule of interest can be
described using these groups exclusively. Higher order groups serve to
distinguish polyfunctional molecules and isomers
\cite{ganiAutomaticCreationMissing2005}.

\citet{matteiModelingCriticalMicelle2013} introduced third-order groups to
improve their model's accuracy by analysing the molecules with the highest
prediction errors after training an initial model with just the default set of
first- and second-order groups.

TODO: Discuss feature selection analogy and metrics. Tie-in with GNN approach by Qin et al.