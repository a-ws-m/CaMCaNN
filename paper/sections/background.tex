Broadly, CMC predictive models take four forms: empirical, semi-empirical,
theoretical and simulated, or some combination of these. Here we will focus on
predictive models for aqueous solutions containing a single surfactant and
discuss some of the trade-offs between the different approaches with regards to
speed, universality and interpretability.

Theoretical approaches have the potential to be the most useful type of
predictive model, if they are accurate and applicable to the desired system, as
they are directly related to scientific knowledge and their results can be
understood in terms of well studied principles.
\citet{puvvadaMolecularThermodynamicApproach1990} derived a phenomenological
model for studying aqueous nonionic surfactant systems that enabled CMC
prediction, as well as modelling of other properties across a range of
temperatures. The model they developed was the product of decomposing the
process of micellisation into discrete steps that they could describe
thermodynamically so as to yield an description of the free energy of
micellisation in terms of a set of molecular parameters:

\begin{itemize}
    \item The tail length, described as the number of carbon atoms.
    \item The average cross-sectional area of the headgroup, which controls the
        steric contribution to the free energy. This must be estimated.
    \item The Tolman length of the tail, which effectively describes the
        thickness of an `interaction region' around the tail
        \cite{demiguelGibbsThermodynamicsSurface2021}. This must also be estimated.
\end{itemize}

A functional form to estimate the parameters was described for linear, nonionic,
polyoxyethylene alcohol surfactants. The model attained impressive accuracy for
some predictions: a root-mean-squared error (RMSE) of approximately
\SI{0.14}{\log \micro M} for the group \ce{C10E_i}, where $i \in [3, 6]$, and
\SI{0.21}{\log \micro M} for the group \ce{C12E_j}, where $j \in [3, 8]$.
However, for other systems, like \ce{C8E6}, the error is much larger. The
authors expect that this inaccuracy is because the model overestimates the CMC
values for systems in which the micelles do not grow.

The connection that this model established between a small set of physically
meaningful properties that can be estimated and emergent properties of
surfactants is extremely useful, especially because it does not explicitly
require fitting to any experimental data. However, the procedures described for
estimating the Tolman length are only applicable for linear hydrocarbon chains;
not the branched case, or for heterogeneous tail groups. Estimating the average
cross-sectional area of the head group may also not be trivial.

Semi-empirical approaches are grounded in theory, but have parameters that are
optimised based on experimental data. \citet{liStudiesUNIQUACSAFT1998} applied a
segment-based UNIQUAC model (s-UNIQUAC) and a SAFT equation of state to predict
CMCs of linear polyoxyethylene alcohols, by first deriving expressions for the
activity coefficient of a surfactant in water. In the s-UNIQUAC model, a
segment-based local-composition model was used, in which the surfactant was
decomposed into segments and the fugacity could be approximated using the fitted
interaction energies between the segments and water. In this case, the segments
used were \ce{C2H4} and \ce{C2H4O}. In the SAFT approach, the surfactant was
treated as a chain of soft-sphere segments in order to first derive the
Helmholtz energy of the solution and from that to derive the fugacity. In this
case, the segments used were \ce{CH2}/\ce{CH3} (these were treated as the same
segment) and \ce{C2H4O}. The interaction energies of the segments were fitted,
as well as parameters of a function describing the soft sphere diameter of a
segment in a chain as a function of the chain length.

Now: Talk about \citet{chengCorrelationCriticalMicelle2005}.

TODO: Talk about geometric, topological and electronic descriptors, as well as
simulation approaches.