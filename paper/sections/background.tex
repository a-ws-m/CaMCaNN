\citet{puvvadaMolecularThermodynamicApproach1990} derived a phenomenological
model for studying aqueous nonionic surfactant systems that enabled prediction
of CMC and other properties across a range of temperatures. They developed a
model for the free energy of micellization, from which the CMC can be
calculated. Their model was parameterized by three properties of the surfactant
molecule:
\begin{itemize}
    \item The tail length, defined as the number of carbon atoms.
    \item The average cross-sectional area of the headgroup, which controls the
          steric contribution to the free energy. This must be estimated.
    \item The Tolman length of the tail, which approximates the thickness of an
          `interaction region' around the tail
          \cite{demiguelGibbsThermodynamicsSurface2021}. This must also be
          estimated.
\end{itemize}

The authors developed a method to estimate parameters for linear, nonionic,
polyoxyethylene alcohol surfactants. The model attained a root-mean-squared
error (RMSE) of approximately \SI{0.14}{\log \micro M} for the group
\ce{C10E_i}, where $i \in (3, 6)$, and \SI{0.21}{\log \micro M} for the group
\ce{C12E_j}, where $j \in (3, 8)$. However, the error is much larger for other
systems, like \ce{C8E6}. The authors ascribed this inaccuracy to their
prediction that the micelles in these systems would not exhibit much anisotropic
growth; their model is better suited for cylindrical or disk-like (bilayer)
micelles.

The connection this model established between a small set of physically
meaningful properties is extremely useful, especially because it does not
explicitly require fitting to any experimental data. However, the procedures
described for estimating the Tolman length are only applicable to linear
hydrocarbon chains, not branched nor heterogeneous tail groups. Estimating the
average cross-sectional area of the head group may also not be trivial.

Greater generalizability and better accuracy can be realized by considering the
process of aggregation by identifying the behavior of individual atoms, or
small groups thereof. Simulation approaches can model the interaction of these
units with each other and derive the potential energy of a configuration
\cite{frenkelUnderstandingMolecularSimulation2001,joshiReviewAdvancementsCoarsegrained2021,filipeMolecularDynamicsSimulations2022}.
For example, molecular dynamics (MD) simulations treat individual atoms, in the
all-atomistic (AA) approach, or groups of atoms, in the coarse-grained (CG)
approach, as particles in a box that interact with each other. This allows the
particles' movement to be simulated by integrating the equations of motion.

For example, \citet{jorgeMolecularDynamicsSimulation2008} used an AA approach to
simulate the self-assembly of \textit{n}-decyltrimethylammonium bromide. They
then estimated the CMC by considering the concentration of `free' surfactants,
i.e. surfactants that were not in micelles, which they defined as an aggregate
containing five or more surfactants. However,
\citet{jusufiExplicitImplicitSolventSimulations2015} criticized the free
surfactant concentration approach for modelling CMC of ionic surfactants in
general. They note that free surfactant concentration above the CMC is highly
dependent on the choice of overall surfactant concentration, especially for
ionic surfactants, which necessitates careful extrapolation to accurately
determine CMCs. As an alternative,
\citet{santosDeterminationCriticalMicelle2016} suggest

% TODO

Coarse-graining, which groups atoms into beads, makes simulating longer time
scales accessible \cite{fitzgeraldMultiscaleModelingNanomaterials2015}. For
example, \citet{vishnyakovPredictionCriticalMicelle2013} used dissipative
particle dynamics (DPD) to model the CMCs of \ce{C8E8},
dodecyldimethylamineoxide (DDAO) and
\textit{N}-decanoyl-\textit{N}-methyl-\textsc{D}-glucamide (MEGA-10). They
proposed a methodology for obtaining parameters to describe the bead
interactions based on known infinite dilution activity coefficients,
$\gamma_\infty$. Their calculated CMC for \ce{C8E8} had an error relative to the
average experimental value equal to \SI{0.07}{\log \micro M}, for DDAO it was
\SI{0.06}{\log \micro M}, and for MEGA-10 it was \SI{0.06}{\log \micro M}.

Although this approach can produce extremely accurate results, it is limited to
compounds that can be parameterized by a known $\gamma_\infty$. Furthermore, the
authors do not discuss parameterizing ionic surfactants. Despite their high
computational cost, both AA and CG simulation approaches can offer very deep
insight into the processes occurring at a molecular level; their results yield a
precise description of the arrangement of molecules in a system, their shape and
aggregation number.

Another approach is the conductor-like screening model (COSMO), which decomposes
the problem by treating a molecule as a cavity with a charged surface in a
solvent that acts as a dielectric continuum \cite{klamtCOSMONewApproach1993}.
The cavity's surface is described by the solvent-accessible surface of the
molecule. The geometry of this surface combined with a segment-wise description
of its polarizing charges can be mapped using density functional theory (DFT).

COSMO for realistic solvation (COSMO-RS) adapts the model for more complex types
of solvent \cite{klamtCOSMORSAlternativeSimulation2010}. Solvents only act like
a dielectric continuum when they are capable of perfectly screening the
COSMO-surface of a solute. COSMO-RS uses statistical mechanics to determine the
probability distribution describing how surface charge densities align between
two molecules. This allows chemical potentials to be determined.

\citet{turchiFirstprinciplesPredictionCritical2022} used COSMO-RS to predict
CMCs by treating a micelle as a separate phase and then considering the
two-phase equilibrium between the micelle and an aqueous phase containing free
surfactants. They modelled the micellar `phase' using two strategies. Their
first strategy was to treat the micellar phase as being equivalent to a bulk,
homogeneous phase of surfactant. The CMC could then be determined by the
equilibrium surfactant concentration in the aqueous phase. The authors argued
that this approximation is more valid as the difference in polarity between the
head and tail of a surfactant is reduced, which was the case for the majority of
nonionic surfactants they considered.

Their second strategy was to consider the micellar phase as a bulk, homogeneous
phase of an oil, whose chemistry was analogous to that of the surfactant's tail
group. They then implemented an iterative procedure to calculate the interfacial
tension (IFT) between the oil and aqueous surfactant phases at different
concentrations of surfactant. The concentration at which the IFT is zero yields
the CMC prediction. The premise of this approach is that the oil phase is
representative of a micelle's interior, which is particularly true as the
interactions between head and tail groups become more unfavorable. This is the
case for highly polar head groups: primarily for ionic surfactants.

The authors recommended applying both strategies and using the lower result as
the CMC prediction. They attained an RMSE of \SI{0.81}{\log \micro M} on a
dataset of 24 surfactants, containing a mix of ionic, nonionic and zwitterionic
surfactants. It is notable that the technique can be applied across all classes
of surfactants.

Other approaches have extended COSMO-RS to explicitly account for the internal
structure of micelles, such as COSMOmic, which treats a micelle as being made of
concentric layers that each have their own surface charge profiles
\cite{klamtCOSMOmicMechanisticApproach2008}. For example,
\citet{jakobtorweihenPredictingCriticalMicelle2017} calculated CMCs using
COSMOmic by first performing MD simulations to attain the layer-wise atomic
distributions. The authors then predicted the CMCs of several polyoxyethylene
alcohols by determining the partition coefficients of inserting the respective
surfactant monomer into a micelle.

COSMOplex is a recent extension of COSMOmic that removes the need to perform and
initial MD simulation to determine the micellar structure
\cite{klamtCOSMOplexSelfconsistentSimulation2019}. Instead, it optimizes the
micellar structure using a self-consistent approach, which iteratively yields
new estimates for the layer-wise charge distributions. The authors predicted the
CMCs of 10 nonionic surfactants with varied head and tail group chemistries,
achieving an RMSE of \SI{0.86}{\log \micro M}.

Although COSMO techniques are promising,
\citet{herbertDielectricContinuumMethods2021} note that many modern extensions
are only available in the proprietary software package \textsc{COSMOTHERM}
\cite{eckertFastSolventScreening2002}.

Another approach to predict CMCs is to use an equation of state. For example,
\citet{liStudiesUNIQUACSAFT1998} applied a segment-based UNIQUAC model
(s-UNIQUAC) and a SAFT equation of state to predict CMCs of linear
polyoxyethylene alcohols by first deriving expressions for the activity
coefficient of a surfactant in water.

\citet{chengCorrelationCriticalMicelle2005} compared the performance of several
models on a large dataset: the polymer-NRTL model
\cite{liStudiesUNIQUACSAFT1998}, a UNIFAC model
\cite{voutsasPredictionCriticalMicelle2001} and a modified Aranovich and Donohue
(m-AD) model \cite{chengCorrelationCriticalMicelle2005}. The predictive
performance of the models on the molecular series \ce{C_nE6}, \ce{C_nE8},
\ce{C_nE9}, \ce{C10E_n} and \ce{C12E_n} were compared, and the resulting RMSEs
are summarized in Table \ref{tab:segment-methods}. The models all have a
reasonably good accuracy, but the SAFT model is particularly good.

\begin{table}
    \caption{Comparison of the RMSEs of selected models on polyoxyethylene
        alcohols' CMCs. The best RMSE is underlined. Data from
        \citet{chengCorrelationCriticalMicelle2005}.}
    \label{tab:segment-methods}
    \begin{tabular}{lr}
        \toprule
        \multicolumn{1}{c}{Model} & \multicolumn{1}{c}{RMSE (\si{\log \micro M})} \\\midrule
        p-NRTL                    & 0.18                                          \\
        s-UNIQUAC                 & 0.14                                          \\
        SAFT                      & \underline{0.06}                              \\
        UNIFAC                    & 0.14                                          \\
        m-AD                      & 0.11                                          \\\bottomrule
    \end{tabular}
\end{table}

Segment-based semi-empirical methods are very promising for predicting CMCs
within a class of surfactants. Their major drawback is that they are only
applicable to molecules that can be decomposed into segments that have trained
parameters. In addition, they must respect the limitations of the theories they
are based upon.

Finally, purely empirical methods have a very heavy reliance on data abundance.
Empirical methods offer a way of making predictions even when a unified theory
is lacking or computationally too demanding. However, without an underlying
theory, their limitations are not well defined and it is possible for the model
to `learn' trends that contradict scientific intuition.

Empirical QSPR methods require validation to determine their reliability and
applicability domain
\cite{veerasamyValidationQSARModelsstrategies2011,tropshaBestPracticesQSAR2010,leonardSelectionTrainingTest2006};
the performance metrics during optimization are not a reliable indicator of
generality, or the performance on new molecules. This is often achieved by
partitioning the available data into \emph{training} and \emph{test} subsets;
the former is used for optimizing the model's parameters, the latter is `hidden'
from the model until training is complete, and the prediction metrics on the
test data indicate how the model can be expected to perform in general. The test
set should span the chemical space in which the model is intended to be applied
\cite{leonardSelectionTrainingTest2006}.

Empirical QSPR models can be used to design novel molecules with target
properties
\cite{gantzerInverseQSPRNovoDesign2020,bolboacaMolecularDesignQSARs2013} and are
interpretable \cite{zefirovFragmentalApproachQSPR2002}, meaning that they can be
analyzed to obtain chemical insights.

\citet{matteiModelingCriticalMicelle2013} extended the Marrero and Gani
group-contribution method \cite{ganiAutomaticCreationMissing2005} to predict the
CMCs of 150 nonionic surfactants. The descriptors are the number of each group
present in a molecule. In the original method
\cite{ganiAutomaticCreationMissing2005}, different `orders' of groups were
identified; the first-order groups are forbidden from overlapping with one
another and they are formulated so that any molecule of interest can be
described using these groups exclusively. Higher order groups distinguish
polyfunctional molecules and isomers \cite{ganiAutomaticCreationMissing2005}.

\citet{matteiModelingCriticalMicelle2013} introduced third-order groups to
improve their model's accuracy by analyzing the molecules with the highest
prediction errors after training an initial model with the first- and
second-order groups from a prior work \cite{ganiAutomaticCreationMissing2005}.
This is an example of \emph{feature selection}, whereby the set of descriptors
is expanded or contracted to adapt to the problem
\cite{liFeatureSelectionData2017,guyonIntroductionVariableFeature2003}.

The authors randomly selected 30 compounds as a test dataset, achieving a RMSE
of \SI{0.13}{\log \micro M}. The model is remarkably accurate and boasts high
interpretability: the fitted contributions of each group describe their effect
on the CMC quantitatively, and the existence of higher order polyfunctional
groups with large contributions implies that their constituent functional groups
have a significant interaction with each other that affects the CMC. However, it
may be difficult to determine whether a new molecule is within the applicability
domain of the model, particularly because positional isomers are not necessarily
distinguished from each other using the group representation.

Recently, an approach based on graph neural networks (GNNs) has produced highly
accurate predictions whilst being applicable to nonionic, cationic, anionic and
zwitterionic surfactants \cite{qinPredictingCriticalMicelle2021}. Neural
networks have many trainable parameters and a complex functional form. This
ensures their versatility as universal approximators but makes them highly
susceptible to overfitting \cite{bejaniSystematicReviewOverfitting2021}. Neural
networks potentially boast the largest applicability domain (for a single set of
trained parameters) of any model discussed previously.

GNN approaches operate on molecular graphs, which are characterized by atomic
nodes whose edges represent bonds. Each operation on this graph considers just
the local environment of an atom, i.e. the atoms that can be reached by
traversing a single bond, but by stacking these operations in sequence, the size
of the environment that is considered increases. In this sense the model is
similar to a group contribution approach, although the groups are determined by
walking $r$ steps along bonds from every atom in the molecule, where $r$ is
equal to the number of subsequent graph operations, so that every group
overlaps. Furthermore, each `contribution' is non-linear, and the number of
contributions is always equal to the number of atoms in a molecule. A more
detailed discussion of GNNs will be given in the Method.