The ideal molecular
representation depends on the task at hand. Ideally, it should be compact but
complete
\cite{faberCrystalStructureRepresentations2015,himanenDScribeLibraryDescriptors2020};
`as simple as possible, but not simpler.' To that end, the representation should
contain enough information to distinguish between isomers that with distinct
properties. However, concessions can be made if we restrict the model's domain
and self-imposed limits on the type of isomers we expose the model to, both
during training and in use. Representations may also include descriptions of
state, such as temperature and pressure \cite{chenGraphNetworksUniversal2019},
but this is redundant in cases where the training data spans a very limited
range of states.

Future efforts to improve this type of model may consider incorporating another
term in the loss function for the GNN that explicitly biases the model towards
learning a form of $\lrv{}$ that captures this similarity. Alternatively, a
variational Gaussian process could be used, which approximates the Gaussian
process using a fixed-size set of `pseudo-points'
\cite{hensmanGaussianProcessesBig2013a}; this would enable the entire GNN/GP
model to be trained at once using backpropagation
\cite{moriartyUnlockNNUncertaintyQuantification2022}.

TODO: Examine interpretability, how easy it is to determine what is within the
applicability domain of the models and the advantages and limitations of the UQ
model.