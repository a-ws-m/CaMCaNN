The ideal molecular
representation depends on the task at hand. Ideally, it should be compact but
complete
\cite{faberCrystalStructureRepresentations2015,himanenDScribeLibraryDescriptors2020};
`as simple as possible, but not simpler.' To that end, the representation should
contain enough information to distinguish between isomers that with distinct
properties. However, concessions can be made if we restrict the model's domain
and self-imposed limits on the type of isomers we expose the model to, both
during training and in use. Representations may also include descriptions of
state, such as temperature and pressure \cite{chenGraphNetworksUniversal2019},
but this is redundant in cases where the training data spans a very limited
range of states.

The representations employed by both the GNN and the linear models capture
topological information and the performances of all of the models on in-domain
data suggest that this is sufficient for the task of predicting CMC very accurately.
However, both of the models are unable to distinguish between certain positional
isomers, depending on the size of the atomic environments that they consider.
In the case of ECFPs, this is dictated by the radius around each atom that
is included in the fingerprint, whilst for the GNNs, this is determined by
the number of consecutive graph layers. Increasing these parameters both
increases computational cost and model complexity, introducing more parameters
and therefore requiring more data in order to optimise them appropriately.

One of the great advantages of using such a topological approach is that the
contributions of each molecular fragment can be explicitly determined, as shown
in the section on ECFP interpretations. In the case of the GNN, introducing a
kernel to the model, via a Gaussian process operating on the GNN's learned
latent space representations, offers a quantitative measure of molecular
similarity that can simultaneously be employed for adding uncertainty to the CMC
predictions and visualising the chemical space of the training data.
Superimposing the test data onto this graph of chemical space highlights which
molecules may have erroneous predictions, based on the fact that they are
clustered amongst training data molecules with very different chemistries.

Future efforts to improve this type of model may consider incorporating another
term in the loss function for the GNN that explicitly biases the model towards
learning a form of $\lrv{}$ that captures this similarity. Alternatively, a
variational Gaussian process could be used, which approximates the Gaussian
process using a fixed-size set of `pseudo-points'
\cite{hensmanGaussianProcessesBig2013a}; this would enable the entire GNN/GP
model to be trained at once using backpropagation
\cite{moriartyUnlockNNUncertaintyQuantification2022}.